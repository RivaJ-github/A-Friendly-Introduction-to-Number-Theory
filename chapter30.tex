\chapter{方程$X^4+Y^4=Z^4$}
\begin{theorem}[指数为4的费马大定理]
方程
\[x^4+y^4=z^2\]
没有整数解$x,y,z$。
\end{theorem}
%
\exercise \proof 假设方程有解$(x,y,z)$。首先证明,如果$x$和$z$有公因子,则可以分解它得到一个更小的解。设$p$是任意整除$\gcd(x,z)$的素数,这意味着$x=pX$,$z=pZ$。代入原方程得$y^2=p^3(X^3+p^2XZ^4)$,这意味着$p^3\mid y^2$,所以$p^2\mid y$,记为$y=p^2Y$。代入原方程,并消去公因子得$pY^2=(X^3+p^2XZ^2)$,这意味着$p\mid X^3$,所以$p\mid X$,设$X=pW$,代入得到$Y^2=p^2(W^3+WZ^4)$。现在$p^2\mid Y^2$,所以$p\mid Y$,设$Y=pV$,代入得$V^2=W^3+WZ^4$。这就得到了原方程得另一个解$(W,V,Z)=(x/p^2, y/p^3, z/p)$,我们消去了$x$,$z$的一个公因子$p$,重复此操作可以得到满足$\gcd(x,z)=1$的一个解。换句话说,如果没有满足$\gcd(x,z)=1$的解,则原式没有解。\par
下面假设原方程有一个满足$\gcd(x,z)=1$的解,考虑$y^2=x(x^2+z^4)$,由于$\gcd(x,z)=1$,所以$\gcd(x,x^2+z^4)=1$。两个互素的数的乘积得到了一个平方数,所以这两个数都是平方数。因此,$x=u^2$,$x^2+z^4=v^2$。将前一个式子代入后一个式子得到$u^4+z^4=v^2$。该式已被证明无解。
%
\exercise a)将$x=y$代入马尔可夫方程:
\[2x^2+z^2=3x^2z\]
所以$x^2\mid z^2$,$x\mid z$。设$z=xw$,得到
\[2+w^2=3xw\]
所以$w\mid 2$,即$w=1$或$w=2$。如果$w=1$,则$z=x=1$,得到解$(1,1,1)$。如果$w=2$,则$x=1$,$z=xw=2$,得到解$(1,1,2)$。\par
b)因为可以随时交换变量,所以只需证明$H(x_0,y_0,z_0)=(x_0,y_0,3x_0y_0-z_0)$是解。观察发现$z=z_0$是下列方程的解:
\[z^2-3x_0y_0z+x_0^2+y_0^2=0\]
方程的两个解的和应为$3x_0y_0$,所以另一个解$z_1$满足$z_1+z_0=3x_0y_0$。换句话说$z=3x_0y_0-z_0$是二次方程的解,所以$(x_0,y_0,3x_0y_0-z_0)$是马尔可夫方程的解。\par
c)略
%
\exercise a)\proof 如果$(x_0,y_0,z_0)$是规范的马尔可夫三元组,只需证明
\[z_0<3x_0z_0-y_0\quad\text{和}\quad z_0<3y_0z_0-x_0\]
即可证明$F(x_0,y_0,z_0)$和$G(x_0,y_0,z_0)$是规范的。首先,
\begin{align*}
y_0&\le z_0 &&\text{$(x_0,y_0,z_0)$是规范的} \\
y_0z_0&\le z_0^2 && \\
y_0z_0& < x_0^2+z_0^2 && x_0^2\ge1 \\
y_0z_0&<3x_0y_0z_0-y_0^2 && \text{由于}x_0^2+y_0^2+z_0^2=3x_0y_0z_0 \\
z_0&<3x_0z_0-y_0 &&
\end{align*}
类似地
\begin{align*}
x_0&\le z_0 &&\text{$(x_0,y_0,z_0)$是规范的} \\
x_0z_0&\le z_0^2 && \\
x_0z_0& < y_0^2+z_0^2 && y_0^2\ge1 \\
x_0z_0&<3x_0y_0z_0-x_0^2 && \text{由于}x_0^2+y_0^2+z_0^2=3x_0y_0z_0 \\
z_0&<3y_0z_0-x_0 &&
\end{align*}
证毕。\par
b)由a)已证
\begin{gather*}
S(x_0,y_0,z_0) < S(F(x_0,y_0,z_0)) \\
S(x_0,y_0,z_0) < S(G(x_0,y_0,z_0)) 
\end{gather*}
还需证明的是$3x_0y_0-z_0<z_0$。$z_0$是下面二次方程的解(注意,$H(1,1,1)=H(1,1,2)$是显然的范例,所以下面排除$x_0=y_0=1$的情况):
\[z^2-3x_0y_0z+x_0^2+y_0^2=0\]
所以
\begin{align*}
2z_0&=3x_0y_0\pm\sqrt{9x_0^2y_0^2-4(x_0^2+y_0^2)} \\
&=3x_0y_0\pm\sqrt{x_0^2y_0^2 + 8x_0^2y_0^2 -4(x_0^2+y_0^2)} \\
&=3x_0y_0\pm\sqrt{x_0^2y_0^2 + 4(x_0^2-1)y_0^2 + 4(y_0^2-1)x_0^2} 
\end{align*}
假设取负号,则
\[2z_0=3x_0y_0-\sqrt{x_0^2y_0^2 + 4(x_0^2-1)y_0^2 + 4(y_0^2-1)x_0^2} < 3x_0y_0-x_0y_0 = 2x_0y_0\]
得到$z_0<x_0y_0$,这将导致
\[x_0^2+y_0^2+z_0^2=3x_0y_0z_0\ge3z_0^2\]
而这是不可能的。因此$z_0$只能取正号。
\[2z_0=3x_0y_0+\sqrt{x_0^2y_0^2 + 4(x_0^2-1)y_0^2 + 4(y_0^2-1)x_0^2} \ge 3x_0y_0+x_0y_0 = 4x_0y_0\]
这比$3x_0y_0-z_0<z_0$更严格。\par
c)\proof 假设存在一些规范马尔可夫三元组不能通过对$(1,1,1)$反复应用$F$和$G$得到。从这些三元组中,我们取长度最小的,记为$(x_0,y_0,z_0)$。根据a)的结论,它的坐标各不相同,否则它将是$(1,1,1)$或$(1,1,2)$,但是$(1,1,2)=F(1,1,1)$。所以我们可以得到$x_0<y_0<z_0$。\par
由b)可知
\[\mathrm{size}(x_0, y_0, 3x_0y_0-z_0)=\mathrm{size}H(x_0,y_0mz_0)>\mathrm{size}(x_0,y_0,z_0)\]
重排$H(x_0,y_0,z_0)$,下面2个马尔可夫三元组中至少一个是规范的:
\[P=(3x_0y_0-z_0,x_0, y_0),\quad Q=(x_0,3x_0y_0-z_0, y_0) \]
由于$P$或$Q$其中之一是规范的。由于我们假设$(x_0,y_0,z_0)$是最小的无法通过$(1,1,1)$得到的规范马尔可夫三元组。所以$P$或$Q$肯定能通过$(1,1,1)$得到。然而
\begin{gather*}
  G(P)=G(3x_0y_0-z_0,x_0, y_0)=(x_0, y_0, 3x_0y_0-(3x_0y_0-z_0))=(x_0, y_0, z_0)\\
  F(Q)=F(x_0,3x_0y_0-z_0, y_0)==(x_0, y_0, 3x_0y_0-(3x_0y_0-z_0))=(x_0, y_0, z_0)
\end{gather*}
即$(x_0, y_0, z_0)$也能通过对$(1,1,1)$反复应用$F$和$G$得到,这与假设矛盾。证毕。