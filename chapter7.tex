\chapter{因数分解与算术基本定理}
\begin{lemma}
令$p$是素数,假设$p$整除乘积$ab$,则$p$整除$a$或$p$整除$b$(或者$p$既整除$a$也整除$b$)。
\end{lemma}
\begin{theorem}[素数整除性质]
假设素数$p$整除乘积$a_1a_2\cdots a_r$,则$p$整除$a_1$,$a_2$,$\cdots$,$a_r$中至少一个因数。
\end{theorem}
\begin{theorem}[素数基本定理]
每个整数$n\ge2$可唯一分解成素数乘积
\[n=p_1p_2\cdots p_r\]
\end{theorem}
%
\exercise \proof $\gcd(a,b)=1$意味着方程$ax+by=1$有整数解。方程两边乘以$c$得到$acx+bcy=c$。显然$a$整除$acx$,又已知$a$整除$bc$。因此$a$整除$c$。
%
\exercise \proof 已知$a$,$b$整除$c$,因此存在整数$a'$,$b'$是的$c=aa'=bb'$。又已知$\gcd(a,b)=1$,因此方程$ax+by=1$有整数解。两边乘以$a'$得到$aa'x+a'by=cx+a'by=bb'x+a'by=a'$。将其代入$c=aa'$得到
\[c=aa'=a(bb'x+a'by)=ab(b'x+a'y)\]
于是$ab$整除$c$,证毕。
%
\exercise \proof 首先某个素数$p$整除$st$,$\frac{s^2-t^2}{2}$,$\frac{s^2+t^2}{2}$中的任意两个,则它同时整除这三个数。因为
\[(st)^2+\left(\frac{s^2-t^2}{2}\right)^2+\left(\frac{s^2+t^2}{2}\right)^2\]
于是$p$整除$\frac{s^2+t^2}{2} + \frac{s^2-t^2}{2} = s^2$,根据素数整除性质,$p$整除$s$。类似的,由于$p$整除$\frac{s^2+t^2}{2} - \frac{s^2-t^2}{2} = t^2$,得到$p$整除$t$。这与已知$\gcd(s,t)=1$矛盾。因此不存在素数$p$能同时整除$st$,$\frac{s^2-t^2}{2}$,$\frac{s^2+t^2}{2}$中的任意两个数。因此这三个数两两互素。
%
\exercise 略,数学归纳法的练习
%
\exercise a)偶数$n$是$\mathbb{E}$-素数,当且仅当$n/2$是奇数。\par
\proof 必要性:如果$n/2$是偶数,那么$n$能被2$\mathbb{E}$-整除,将不是$\mathbb{E}$-素数。\par
充分性:如果$n/2$是奇数,那么对于任意能整除$n$的偶数$m$,$n/m$必是奇数。因此不存在能$\mathbb{E}$-整除$n$的数。即$n$是$\mathbb{E}$-素数。\par
b)如果$n$是$\mathbb{E}$-素数,则得证。否则,$n=n_1n_2$,将$n_1$,$n_2$中的$\mathbb{E}$-素数放在一边。否则分解他们。这个过程必定可以终止,因为每次分解的因子必定更小。因此,我们总能将$n$写成$\mathbb{E}$-素数的乘积。\par
c)有两种不同分解法分解成$\mathbb{E}$-素数乘积的最小的数是36,$36=2\times18=6\times6$。\par
能写成$4p^2q$(其中$p$,$q$是不同的奇素数)的数有三种不同分解法,$2\cdot2p^2q$,$2p\cdot 2pq$,$2p^2\cdot 2q$。因此最小的数就是$4\times3^2\times5=180$。\par
能写成$4pqr$(其中$p$,$q$,$r$是不同的奇素数)的数有四种不同分解法,$2\cdot 2pqr$,$2p\cdot 2qr$,$2q\cdot 2pr$,$2r\cdot 2pq$,其中最小的数是$4\times3\times5\times7=420$。另一种有四种不同分解法的形式是$4p^3r$,其中最小的是$4\times3^3\times5=540$。然而还有一个形式$8p^2q$,其分解方法为$2\cdot 2\cdot 2p^2q$,$2\cdot 2p\cdot 2pq$,$2\cdot 2p^2\cdot 2q$,$2p\cdot 2p\cdot 2q$,其中最小的是$8\times3^2\times5=360$\par
d)有三种情况会产生仅有一种分解法的偶数。1、它本身就是$\mathbb{E}$-素数;2、它没有奇因子;3、它仅有一个奇因子。总结如下,一个偶数$n$仅有一种$\mathbb{E}$-素数分解法,当且仅当$n$满足如下形式之一:
\begin{align*}
n&=2k  \text{\quad $k$是奇数} \\
n&=2^r \text{\quad $r\ge 1$} \\
n&=2^rp \text{\quad $p$为素数且$r\ge2$}
\end{align*}
%
\exercise a)5,9,13,17,21,29\par
b)如果$p$和$q$是模4余3的素数,则$pq$是$\mathbb{M}$-素数。因为此时$pq\equiv 1\pmod 4$,但$pq$无法被分解为两个模4余1的数的乘积。据此,我们可以构造形如$p^2q^2$的$\mathbb{M}$-素数。如$441=3^2\cdot7^2=9\cdot49=21\cdot21$。也可以构造形如$p^2qr$的$\mathbb{M}$-素数。如$693=3^2\cdot7\cdot11=9\cdot77=21\cdot33$
%
\exercise 代码详见code/exe\_7\_7.py。其中方法a)运行很慢;优化后的方法b)运行速度提升近1000倍。其中提前准备素数列表提升了近2倍。利用$m$不能倍2与$\sqrt{m}$见的任何数整除时$m$必为素数的事实提升了近500倍。\par
\begin{tabular}{ll}
    \centering
    $1000000=2^{6} \cdot 5^{6}$ & $1000016=2^{6} \cdot 5^{6}$ \\
    $1000001=101 \cdot 9901$ & $1000017=101 \cdot 9901$ \\
    $1000002=2 \cdot 3 \cdot 166667$ & $1000018=2 \cdot 3 \cdot 166667$ \\
    $1000003=1000003$ & $1000019=1000003$ \\
    $1000004=2^{2} \cdot 53^{2} \cdot 89$ & $1000020=2^{2} \cdot 53^{2} \cdot 89$ \\
    $1000005=3 \cdot 5 \cdot 163 \cdot 409$ & $1000021=3 \cdot 5 \cdot 163 \cdot 409$ \\
    $1000006=2 \cdot 7 \cdot 71429$ & $1000022=2 \cdot 7 \cdot 71429$ \\
    $1000007=29 \cdot 34483$ & $1000023=29 \cdot 34483$ \\
    $1000008=2^{3} \cdot 3^{2} \cdot 17 \cdot 19 \cdot 43$ & $1000024=2^{3} \cdot 3^{2} \cdot 17 \cdot 19 \cdot 43$ \\
    $1000009=293 \cdot 3413$ & $1000025=293 \cdot 3413$ \\
    $1000010=2 \cdot 5 \cdot 11 \cdot 9091$ & $1000026=2 \cdot 5 \cdot 11 \cdot 9091$ \\
    $1000011=3 \cdot 333337$ & $1000027=3 \cdot 333337$ \\
    $1000012=2^{2} \cdot 13 \cdot 19231$ & $1000028=2^{2} \cdot 13 \cdot 19231$ \\
    $1000013=7 \cdot 373 \cdot 383$ & $1000029=7 \cdot 373 \cdot 383$ \\
    $1000014=2 \cdot 3 \cdot 166669$ & $1000030=2 \cdot 3 \cdot 166669$ \\
    $1000016=2^{4} \cdot 62501$ & \\
    \end{tabular}