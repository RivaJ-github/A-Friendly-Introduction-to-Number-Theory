\chapter{二次互反律的证明}
\begin{theorem}[高斯准则]
设$p$为奇素数,$a$是不被$p$整除的整数,记$P=\frac{p-1}{2}$,
\[\mu(a,p)=\left({a,2a,3a,\cdots, Pa\text{模$p$简化到$-P$与$P$}\atop \text{之间后变为负数的那些整数的个数}}\right)\]
则
\[\left(\frac{a}{p}\right)=(-1)^{\mu(a,p)}\]
\end{theorem}
\begin{lemma}
当$a, 2a, 3a,\cdots,Pa$模$p$简化到$-P$与$P$之间时,简化后的值时$\pm1,\cdots,\pm P$按某种顺序的排列,每个数以带正号的形式或带负号的形式出现一次。
\end{lemma}
\begin{lemma}
设$p$为奇素数,令$P=\frac{p-1}{2}$,$a$是不被$p$整除的奇数,则有
\[\sum_{k=1}^P \left\lfloor\frac{ka}{p}\right\rfloor\equiv\mu(a,p)\pmod2\]
\end{lemma}
%
\exercise a)$\left\lfloor-\frac{7}{3}\right\rfloor=\lfloor-2.333\rfloor=-3$;b)$\left\lfloor\sqrt{23}\right\rfloor=\lfloor4.79\dots\rfloor=4$;c)$\left\lfloor\pi^{2}\right\rfloor=\lfloor9.86\dots\rfloor=9$;d)$\left\lfloor\frac{\sqrt{73}}{\sqrt[3]{19}}\right\rfloor=\lfloor3.20\dots\rfloor=3$
%
\exercise a)$n$是整数,$x$是实数,则根据定义有\[\lfloor x+n\rfloor=\lfloor x\rfloor+n\]
于是
\begin{align*}
f(x+n)&=\lfloor2(x+n)\rfloor-2\lfloor x+n\rfloor \\
&=\lfloor2x+2n\rfloor-2\lfloor x+n\rfloor \\
&=(\lfloor 2x\rfloor+2n)-2(\lfloor x\rfloor+n) \\
&=\lfloor2x\rfloor-2\lfloor x\rfloor \\
&=f(x)
\end{align*}
b)\[f(x)=\begin{cases}
0 & 0\le x<\frac{1}{2} \\
1 & \frac{1}{2}\le x<1
\end{cases}\]
根据a)的结论,对任意实数$x$,都有$f(x)=0$或1。因为总可以将$x$表示为$x=n+t$,其中$n$是整数,$t$是实数,且$0\le t<1$。所以$f(x)=f(t)$,而我们的猜想适用于$f(t)$。\par
c)\proof 若$0\le x<\frac{1}{2}$,则$0\le 2x<1$,所以
\[f(x)=\lfloor 2x\rfloor-2\lfloor x\rfloor=0-0=0\]
若$\frac{1}{2}\le x<1$,则$1\le 2x<2$,所以
\[f(x)=\lfloor 2x\rfloor-2\lfloor x\rfloor=1-0=1\]
%
\exercise a)
\begin{center}
\begin{tabular}{rrrr}
$g(0) = 0$ & $g(0.25)=0$ & $g(0.5)=1$ &  \\
$g(1) = 2$ & $g(2)=4$ & $g(2.5)=5$ & $g(2.49)=4$ \\
\end{tabular}
\end{center}
b)$g(x)=\lfloor 2x\rfloor$\par
c)\proof 记$x=n+t$,其中$n$是整数,$0\le t<1$。如果$0\le t<\frac{1}{2}$,则
\[g(x)=\lfloor n+t\rfloor+\left\lfloor n+t+\frac{1}{2}\right\rfloor=n+n=2n=2n+\lfloor2t\rfloor=\lfloor 2x\rfloor\]
如果$\frac{1}{2}\le t<1$,则
\[g(x)=\lfloor n+t\rfloor+\left\lfloor n+t+\frac{1}{2}\right\rfloor=n+n+1=2n+1=2n+\lfloor2t\rfloor=\lfloor 2x\rfloor\]
d)
\[g(x)=\lfloor x\rfloor\left\lfloor x+\frac{1}{3}\right\rfloor+\left\lfloor x+\frac{2}{3}\right\rfloor=\lfloor3x\rfloor\]
证明过程类似c)
e)\[g(x)=\lfloor x\rfloor\left\lfloor x+\frac{1}{N}\right\rfloor+\left\lfloor x+\frac{2}{N}\right\rfloor+\cdots + \left\lfloor x+\frac{N-1}{N}\right\rfloor=\lfloor Nx\rfloor\]
\proof 将$x$分解为$x=n+t$并选择$0\le k<N$,考虑$k/N\le t<(k+1)/N$的场景。
\begin{align*}
\sum_{i=0}^{N-1}\left\lfloor x+\frac{i}{N}\right\rfloor&=\sum_{i=0}^{N-1}\left\lfloor n+t+\frac{i}{N}\right\rfloor=\sum_{i=0}^{N-1}n+\left\lfloor t+\frac{i}{N}\right\rfloor \\
&=nN + \sum_{i=0}^{N-1}\left\lfloor t+\frac{i}{N}\right\rfloor
\end{align*}
注意到
\[\frac{k+i}{N}\le t+\frac{i}{N}<\frac{k+i+1}{N}\]
所以
\[\left\lfloor t+\frac{i}{N}\right\rfloor=
\begin{cases}
0 & 0\le i \le N-k-1 \\
1 & N-k\le i \le N-1
\end{cases}\]
因此
\[\sum_{i=0}^{N-1}\left\lfloor x+\frac{i}{N}\right\rfloor=nN+\sum_{i=N-k}^{N-1}1=nN+k\]
另一方面
\[\lfloor Nx\rfloor=\lfloor Nn+Nt\rfloor=Nn+\lfloor Nt\rfloor\]
由于$k/N\le t<(k+1)/N$,所以$\lfloor Nt\rfloor=k$,所以
\[\lfloor Nx\rfloor=Nn+k\]
%
\exercise a)设$ka=q_kp+r_k$,已知$a$是偶数$p$是奇数,则对其模2得到
\[0\equiv q_k+r_k\pmod 2\]
对其求和可得
\[0\equiv\sum_{k=1}^P q_k+\sum_{k=1}^P r_k\pmod 2\]
引理23.2告诉我们$r_1,\cdots,r_p$是$\pm1,\cdots,\pm P$的某个排列。模2操作下可以忽略其符号。所以
\[\sum_{k=1}^P r_k\equiv 1+2+\cdots + P\pmod 2\]
对其求和:
\[1+2+\cdots + P=\frac{P(P+1)}{2}=\frac{\frac{p-1}{2}\cdot\frac{p+1}{2}}{2}=\frac{p^2-1}{8}\]
联立前面三式:
\[\sum_{k=1}^P q_k\equiv\sum_{k=1}^P r_k\equiv\sum_{k=1}^P k\equiv\frac{p^2-1}{8}\pmod2\]
在引理23.3的证明过程中,我们得到了
\[\sum_{k=1}^P\left\lfloor\frac{ka}{p}\right\rfloor=\sum_{k=1}^P q_k-\mu(a,p)\]
联立最后两式即得证。\par
b)考虑
\[\sum_{k=1}^P\left\lfloor\frac{2k}{p}\right\rfloor\]
当$1\le k\le P$时,$2\le 2k\le p-1$,所以$\lfloor\frac{2k}{p}\rfloor=0$。因此和式为0。根据a)的结论:
\[\mu(2,p)\equiv \frac{p^2-1}{8}\pmod2\]
由高斯准则
\[\left(\frac{2}{p}\right)=(-1)^{\mu(2, p)}=(-1)^{\frac{p^2-1}{8}}\]
%
\exercise a)$A=\frac{15}{2}$;$N=4$;$B=9$;$A-N-\frac{1}{2}B=-1$\par
b)$A=12$;$N=7$;$B=12$;$A-N-\frac{1}{2}B=-1$\par
c)显然可以猜想$A-N-\frac{1}{2}B=-1$\par
d)\proof 显然位于$x$轴上的整数点有$a+1$个,位于竖线$x=b$上的整数点有$b+1$个。设$d=\gcd(a,b)$,则位于斜边上的点有$d+1$个,分别为
\[(0,0),\left(\frac{a}{d}, \frac{b}{a}\right),\left(\frac{2a}{d}, \frac{2b}{a}\right),\cdots,\left(\frac{(d-1)a}{d}, \frac{(d-1)b}{a}\right)\]
注意我们重复计算了三个顶点$(0,0),(a,0),(a,b)$。所以
\[B=a+b+d,\qquad \text{其中}d=\gcd(a,b)\]
面积$A=\frac{1}{2}ab$。\par
构造一个矩形$(0, 0),(a,0),(0,b),(a,b)$,它将包括两个三角形。矩形内部有$(a-1)(b-1)$个点。这些内部点中,有$d-1$个点在对角线上。矩形内部不在对角线也不在边上的点的个数为:
\[2N=(a-1)(b-1)-(d-1)\]
现在可以计算:
\[A-N-\frac{1}{2}B=\frac{1}{2}ab-\frac{(a-1)(b-1)-(d-1)}{2}-\frac{1}{2}(a+b+d)=-1\]