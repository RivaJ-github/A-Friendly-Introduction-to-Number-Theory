\chapter{素数的计数}
\begin{theorem}[素数定理]
当$x$很大时,小于$x$的素数个数近似等于$x/\ln x$,换句话说,
\[\lim_{x\rightarrow\infty}\frac{\pi(x)}{x/ln x}\]
\end{theorem}
\begin{conjecture}[哥德巴赫猜想]
每个偶数$n\ge4$可表成两个素数之和。
\end{conjecture}
\begin{conjecture}[孪生素数猜想]
存在无穷多个素数$p$使得$p+2$也是素数。
\end{conjecture}
\begin{conjecture}[$N^2+1$猜想]
存在无穷多个形如$N^2+1$的素数。
\end{conjecture}
%
\exercise a)当$x$很大时,$F(x)/x$约为1/5。事实上,$F(x)$总满足
\[\frac{x}{5}-1\le F(x)\le\frac{x}{5}\]
所以当$x$很大时,$F(x)/x$和1/5的差最多为$1/x$。\par
b)$S(x)$近似于$\sqrt x$。
%
\exercise a)
\begin{align*}
70&=3+67=11+59=17+53=23+47=29+41 \\
72&=5+67=11+61=13+59=19+53=29+43=31+41 \\
74&=3+71=7+67=13+61=31+43=37+37 \\
76&=3+73=5+71=17+59=23+53=29+47 \\
78&=5+73=7+71=11+67=17+61=19+59=31+47=37+41 \\
80&=7+73=13+67=19+61=37+43 \\
82&=3+79=11+71=23+59=29+53=41+41 \\
84&=5+79=11+73=13+71=17+67=23+61=31+53=37+47=41+43 \\
86&=3+83=7+79=13+73=19+67=43+43 \\
88&=5+83=17+71=29+59=41+47 \\
90&=7+83=11+79=17+73=19+71=23+67=29+61=31+59=37+53=43+47 \\
92&=3+89=13+79=19+73=31+61 \\
94&=5+89=11+83=23+71=41+53=47+47 \\
96&=7+89=13+83=17+79=23+73=29+67=37+59=43+53 \\
98&=19+79=31+67=37+61 \\
100&=3+97=11+89=17+83=29+71=41+59=47+53 
\end{align*}
b)见a)
%
\exercise 对任意$2\le k\le n$,$k\mid n!$,所以$k\mid n!+k$,所以$n!+k$是合数。
%
\exercise 猜想但仍未证明存在形如a)、b)、d)的无穷多个素数。但是对于c),由于$N^2+3N+2=(N+1)(N+2)$,所以不存在无限多个这样的素数。
%
\exercise 略
%
\exercise a)由于$\int_2^xdt/\ln(t)$和$x/\ln(x)$在$x\rightarrow \infty$时趋于无穷,所以可以应用洛必达法则:
\[\lim_{x\rightarrow\infty}\frac{\int_2^x\frac{dt}{\ln(t)}}{\frac{x}{\ln(x)}}=\lim_{x\rightarrow\infty}\frac{\frac{d}{dx}\int_2^x\frac{dt}{\ln(t)}}{\frac{d}{dx}\frac{x}{\ln(x)}}\]
由微积分第二基本定理$\frac{d}{dx}\infty_a^x f(t)dt=f(x)$,所以
\begin{align*}
\lim_{x\rightarrow\infty}\frac{\int_2^x\frac{dt}{\ln(t)}}{\frac{x}{\ln(x)}}&=\lim_{x\rightarrow\infty}\frac{\frac{1}{\ln(x)}}{\frac{\ln(x)-x(1/x)}{(\ln(x))^2}} \\
&=\lim_{x\rightarrow\infty}\frac{\ln(x)}{\ln(x)-1} \\
&=1
\end{align*}
b)见code/exe\_13\_6.py,制表如下:
\begin{center}
\begin{tabular}{|c||*{6}{c|}}
\hline
$x$ & 10 & 100 & 1000 & $10^4$ & $10^6$ & $10^9$ \\
\hline
$\pi(x)$ & 4 & 25 & 168 & 1229 & 78498 & 50847534 \\
\hline
$x/\ln(x)$ & 4.34 & 21.71 & 144.76 & 1085.74 & 72382.41 & 48254942.43 \\
\hline
$\int_2^x dt/\ln(t)$ & 5.12 & 29.08 & 176.56 & 1245.09 & 78626.50 & 50849233.91 \\
\hline
\end{tabular}
\end{center}
显然积分方式近似得更好。\par
c)记
\[F(t)=\ln(\ln (t))+\ln(t)+\frac{(\ln(t))^2}{2\cdot2!}+\frac{(\ln(t))^3}{3\cdot3!}+\frac{(\ln(t))^4}{4\cdot4!}+\cdots\]
则
\begin{align*}
F'(t)&=\frac{1}{t\ln(t)}+\frac{1}{t}+\frac{\ln(t)}{t\cdot2!}+\frac{(\ln(t))^2}{t\cdot3!}+\frac{(\ln(t))^3}{t\cdot4!}+\cdots \\
&=\frac{1}{t\ln(t)}+\frac{1}{t}\left(1+\frac{\ln(t)}{2!}+\frac{(\ln(t))^2}{3!}+\frac{(\ln(t))^3}{4!}+\cdots\right)
\end{align*}
由于$e^X$的泰勒展开
\[e^X=1+X+\frac{X^2}{2!}+\frac{X^3}{3!}+\cdots\]
移项除以$X$得到导数方程中大括号内的形势:
\[\frac{e^X-1}{X}=1+\frac{X}{2!}+\frac{X^2}{3!}+\frac{X^3}{4!}+\cdots\]
于是
\begin{align*}
F'(t)&=\frac{1}{t\ln(t)}+\frac{1}{t}\left(\frac{e^{\ln(t)}-1}{\ln(t)}\right) \\
&=\frac{1}{t\ln(t)}+\frac{t-1}{t\ln(t)} \\
&=\frac{1}{\ln(t)} \\
\end{align*}
