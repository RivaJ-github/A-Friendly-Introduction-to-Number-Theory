\chapter{因数分解与算术基本定理}
\begin{theorem}[线性方程定理]{l}
设$a$与$b$是非零整数,$g=\gcd(a,b)$。方程
\[ax+by=g\]
总是有一个整数解$(x_1, y_1)$,它可由欧几里得算法得到。则方程的每一个解可由
\[\left(x_1+k\cdot\frac{b}{g}, y_1-k\cdot\frac{a}{g}\right)\]
得到,其中$k$是任意整数。
\end{theorem}
%
\exercise a)$12345\cdot 11 - 67890\cdot 2=\gcd(12345, 67890) = 15$\par
b)$-54321\cdot 1645 + 9876\cdot 9048 = \gcd(54321, 9876) = 3$
%
\exercise 
a)$105\cdot(-53)+121\cdot46=1$。一般解为$(-53+121k, 46-105k)$\par
b)$12345\cdot11+67890\cdot(-2)=15$。一般解为$(11+4526k, -2-823k)$\par
c)$54321\cdot(-1645)+9876\cdot9048=3$。一般解为$(-1645+3292k, 9048-18107k)$
%
\exercise 
a){\bf 提示:}每次迭代后$w$是新产生的余数,$g$上一次迭代产生的余数。$v$是形如$r_i = ax + by$中的$x$。而$x$则记录的是$r_{i-1} = ax + by$中的$x$。\par
b)python实现如下:
\begin{lstlisting}
def gcdWithXY(a, b):
    x, g, v, w = 1, a, 0, b
    while (w != 0):
        t = g % w
        q = g // w
        s = x - q * v
        x, g = v, w
        v, w = s, t
    return (g, x, (g - a * x) // b)
\end{lstlisting}
c)运行code/exe6\_3.py。
\begin{gather*}
\gcd(19789, 23548) = 7\quad (x,y)=(1303,-1095) \\
\gcd(31875, 8387) = 1\quad (x,y)=(-381,1448) \\
\gcd(22241739, 19848039) = 237\quad (x,y)=(-8980,10063)
\end{gather*}
d)添加特殊处理:
\begin{lstlisting}
    if (b == 0):
        return (a, 1, 0)
\end{lstlisting}
e)在返回前添加校验:
\begin{lstlisting}
    while (x <= 0):
        x += b
\end{lstlisting}
%
\exercise a)首先解$6X+15Y=\gcd(6, 15)=3$,其中一个解是$X=3$,$Y=-1$。
接下来解$3W+20Z=1$。其中一个解是$W=7$,$Z=-1$。即方程
\[(6X+15Y)W + 20Z = 1\]
有解$X=3$,$Y=-1$,$W=7$,$Z=-1$。将方程变形为:
\[6XW+15YW + 20Z = 1\]
因此,原方程有一个解为$x=XW=21$,$y=YW=-7$,$z = Z=-1$。\par
b)$ax+by+cz=1$有整数解,当且仅当$a$,$b$,$c$只有一个公因子1。\par
\proof 必要性:假设$a$,$b$,$c$有公因子$d$,则$d$整除$ax+by+cz$,同时$d$整除1,所以$d$只能是1。\par
充分性:如果$a$,$b$,$c$没有公因子,可通过如下步骤得到一个解。首先解$aX+bY=g=\gcd(a,b)$。因为$a$,$b$,$c$没有公因子,所以$g$和$c$没有公因子。因此可以找到$gW+cZ=1$的解。于是
\[(aX+bY)W +cZ=1\]
最终我们得到解$x=XW$,$y=YW$,$z=Z$\par
c)首先,解$155X+341Y=\gcd(155, 341)=31$,得到一个较小的解$X=-2$,$Y=1$。接下来解$31W+385Z=1$。得到一个解$W=-149$,$Z=12$。于是得到原式的一个解$x=298$,$y=-149$,$z=12$。
%
\exercise \proof 已知$ax+by=\gcd(a,b)=1$比有整数解,记为$x=X$,$y=Y$,即$aX+bY=1$,两边乘以$c$得到$acX+bcY=c$。因此原式有解$x=cX$,$y=cY$\par
解$37x+47y=\gcd(37, 47)=1$,得到$x=14$,$y=-11$,所以$(x,y)=(14*103, -11*103)=(1442, -1133)$是原式的解。于是$(1442 - 47k, -1133+37k)$也是原式的解。令$k=31$,得到一个较小的解$(-15, 14)$
%
\exercise a)(原书印刷错误了)解释为什么方程$3x+5y=4$没有整数解$x\ge0$与$y\ge 0$\par
\proof 如果$y\ge1$则$3x+5y\ge5$原式无解。所以$y=0$,但此时$3x+5y=3x$,$3x=4$没有整数解。综上,原式无整数解。\par
b)不可能出现的数值是$\{1, 2, 4, 7\}$\par
\proof 首先8到15是可行的:$8=3\cdot1+5\cdot1$,$9=3\cdot3+5\cdot1$,$10=3\cdot0+5\cdot2$,$11=3\cdot2+5\cdot1$,$12=3\cdot4+5\cdot0$,$13=3\cdot1+5\cdot2$,$14=3\cdot3+5\cdot1$,$15=3\cdot5+5\cdot0$。对任意$c\ge 16$,都可以写成$c=8d+e$,其中$8\le e<16$。已经证明$e=3u+5v$,所以$c=(3+d)u+(5+d)v$。因此,不可能出现的数值小于8,按a)中的方法枚举即可。\par
c)(\romannumeral1)11;(\romannumeral2)23;(\romannumeral3)29\par
d)如果$\gcd(a, b)=1$,则最大的不能被表示成$ax+by$形式的值是$ab-a-b$。\par
e)严格的证明过程超出了当前的能力,下面给出两个证明的片段:\par
\proof{\bf 不能表示$ab - a - b$}\par
假设$ax+by=ab-a-b$,则$a(x-b+1)=b(-y-1)$。因为$\gcd(a,b)=1$,所以$a$整除$-y-1$。因此对于某个$k\ge1$,$y=ka-1$(因为$y\ge0$,$k$必须$\ge1$)。类似地,因为$b(y-a+1)=a(-x-1)$,因此对于某个$j\ge1$,$x=jb-1$。于是
\begin{align*}
ab-a-b &= ax+by \\
&= ax+by \\
&= a(jb-1)+b(ka-1) \\ 
&= (j+k)ab - a - b \\
\ge 2ab-a-b
\end{align*}
由此得到$0\ge ab$,这与题设矛盾。因此$ax+by=ab-a-b$没有满足$x\ge0$和$y\ge0$的整数解。\par
\proof{\bf 能表示$ab - a - b + 1$}\par
首先$au-bv=1$有整数解,$(u-kb, v+ka)$,我们可以找到一个解使得,$1\le u<b$。这意味着$v=(au-1)/b<au/b<a$。于是得到原式的一个解
\[ab-a-b+1=ab-a-b+au-bv=a(u-1)+b(a-v-1)\]
因为$u\ge1$,$v\le a$,所以$ax+by=ab-a-b+1$存在$x\ge0$,$y\ge0$的整数解。\par
f)对于三项及以上的变量,问题将变得非常苦难。code/exe\_6\_6.py给出了1000以内不能表示为$6x+10y+15z$的整数。
\[\{1, 2, 3, 4, 5, 7, 8, 9, 11, 13, 14, 17, 19, 23, 29\}\]
因此猜测最大符合条件的整数是29。