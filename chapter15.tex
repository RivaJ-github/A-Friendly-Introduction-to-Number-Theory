\chapter{梅森素数与完全数}
\begin{theorem}[欧几里得完全数公式]
如果$2^p-1$是素数,则$2^{p-1}(2^p-1)$是完全数。
\end{theorem}
\begin{theorem}[欧拉完全数定理]
如果$n$是偶完全数,则$n$形如
\[n=2^{p-1}(2^p-1)\]
其中$2^p-1$是梅森素数。
\end{theorem}
\begin{theorem}[$\sigma$函数公式]
(a)如果$p$是素数,$k\le1$,则
\[\sigma(n)=1+p+p^2+...+p^k=\frac{p^{k+1}-1}{p-1}\]
(b)如果$\gcd(m,n)=1$,则
\[\sigma(mn)=\sigma(m)\sigma(n)\]
\end{theorem}
%
\exercise 设$d_1,\cdots,d_r$是能整除$m$的数,$e_1\cdots e_s$是能整除$n$的数。由于$m$和$n$互素,所以通过任取$d_i,e_j$的乘积,可以得到所能整除$mn$的数,即
\[\sigma(mn)=\prod d_ie_j=(d_1+\cdots+d_r)(e_1+\cdots+e_s)=\sigma(m)\sigma(n)\]
%
\exercise 可以使用code/exe\_15\_3计算任意$\sigma$函数的值。\par
a)$\sigma(10)=18$,b)$\sigma(20)=42$,c)$\sigma(1728)=5080$
%
\exercise a、b)
\[\sigma(p^k)=\frac{p^{k+1}-1}{p-1}<\frac{p^{k+1}}{p-1}=\left(\frac{p}{p-1}\right)p^k\]
由于$p>2$时$p/(p-1)<2$,所以$\sigma(p^k)<2p^k$,因此如果$p$是奇素数,则$p^k$不可能是完全数。\par
c、d)
\begin{align*}
\sigma(3^ip^j)&=\sigma(3^i)\sigma(p^j) \\
&=\left(\frac{3^{i+1}-1}{3-1}\right)\left(\frac{p^{i+1}-1}{p-1}\right) \\
&<\left(\frac{3^{i+1}}{3-1}\right)\left(\frac{p^{i+1}}{p-1}\right) \\
&=\left(\frac{3p}{2(p-1)}\right)3^ip^j
\end{align*}
当$p>4$时,$3p/(2p-2)<2$,$\sigma(3^ip^j)<2\cdot3^ip^j$,证毕。\par
e)
\begin{align*}
\sigma(q^ip^j)&=\sigma(q^i)\sigma(p^j) \\
&=\left(\frac{q^{i+1}-1}{q-1}\right)\left(\frac{p^{i+1}-1}{p-1}\right) \\
&<\left(\frac{q^{i+1}}{q-1}\right)\left(\frac{p^{i+1}}{p-1}\right) \\
&=\left(\frac{qp}{(q-1)(p-1)}\right)q^ip^j
\end{align*}
假设$q^ip^j$是完全数,则$2<qp/((q-1)(p-1))$,交叉相乘并简化得到$qp-2q-2p-2<0$,变形可得$(q-2)(p-2)<2$,由于$p$,$q$是不同的奇素数,所以这是不可能的。即$p^iq^j$不可能是完全数。
%
\exercise 在练习15.3,已经证明了单个奇素数的幂和两个奇素数的幂的乘积的情况。因此,这里只需证明$m\ge1$且$n\ge1$且$k\ge1$的情况下,下式不可能成立。
\begin{gather*}
\sigma(3^m\cdot5^n\cdot7^k)=\left(\frac{3^{m+1}-1}{3-1}\right)\left(\frac{5^{n+1}-1}{5-1}\right)\left(\frac{7^{k+1}-1}{7-1}\right)=2\cdot3^m5^n7^k \\
(3\cdot3^m-1)(5\cdot5^n-1)(7\cdot7^k-1)=96\cdot3^m\cdot5^n\cdot7^k
\end{gather*}
由于不能判定等式两边的大小关系,我们需要分情况讨论:\par
1、$m=1$:\par
原式转化为$(5\cdot5^n-1)(7\cdot7^k-1)=36\cdot3^m\cdot5^n\cdot7^k$,这是不可能的,因为左边肯定小于右边。\par
2、$m\ge 2$,$n=k=1$:\par
原式转化为$12(3\cdot3^m-1)=35\cdot3^m$,由于左侧是偶数,右侧是奇数,也不可能成立。所以$n$和$k$至少有一个大于1。\par
3、$m,n\ge2$,$k\ge1$:\par
由$p\cdot p^l-1=(p-\frac{1}{p^l})p^l$,可以得到
\[3\cdot3^m-1\ge\frac{26}{9}3^m,\quad 5\cdot5^n-1\ge\frac{124}{25}5^n,\quad 7\cdot7^k-1\ge\frac{48}{7}7^k\]
因此
\begin{align*}
(3\cdot3^m-1)(5\cdot5^n-1)(7\cdot7^k-1)&\ge \frac{26}{9}3^m\frac{124}{25}5^n\frac{48}{7}7^k \\
&=\frac{51584}{525}\cdot3^m\cdot5^n\cdot7^k\\
&>96\cdot3^m\cdot5^n\cdot7^k
\end{align*}
这种情况也不成立。\par
4、$m,k\ge2$,$n\ge1$:\par
同理可得
\[3\cdot3^m-1\ge\frac{26}{9}3^m,\quad 5\cdot5^n-1\ge\frac{24}{5}5^n,\quad 7\cdot7^k-1\ge\frac{342}{49}7^k\]
因此
\begin{align*}
(3\cdot3^m-1)(5\cdot5^n-1)(7\cdot7^k-1)&\ge \frac{26}{9}3^m\frac{24}{5}5^n\frac{342}{49}7^k \\
&=\frac{23712}{245}\cdot3^m\cdot5^n\cdot7^k\\
&>96\cdot3^m\cdot5^n\cdot7^k
\end{align*}
至此,我们已经讨论了所有的情况,$3^m\cdot5^n\cdot7^k$不可能是完全数。
%
\exercise 任意整数可以表示成素数幂的积,令$n=\prod p_i^{k_i}$,于是
\begin{align*}
\sigma(n^2)&=\prod\sigma(p_i^{2k_i}) \\
&=\prod(p_i^{2k_i}+p_i^{2k_i-1}+\cdots+p_i+1)
\end{align*}
由于每一个乘积项都有奇数项,无论$p_i$是偶数还是奇数,每个乘积项都是奇数。所以$\sigma(n^2)$是奇数。但要使$n^2$是完全数,则$\sigma(n^2)=2n^2$需要是偶数。所以平方数不可能是完全数。
%
\exercise a)6,8,10,14,15,21,22,26,27,33,34,35,38,39,46\par
b)积完全数必须形如$p^3$或$pq$,其中$p$和$q$是不同的素数。\par
c)\proof 假设积完全数$m$被(至少)两个不同的素数$p$和$q$整除。则$m$至少被$m/p$和$m/q$整除。则
\[m=(\text{整除$m$的数的乘积})\ge\frac{m}{p}\cdot\frac{m}{q}=\frac{m^2}{pq}\]
整理得到$pq\ge m$,但是$p$和$q$也整除$m$,所以$pq\le m$。唯一可能的情况是$pq=m$。这说明如果积完全数$m$被两个及以上的不同素数整除,则它只能是$pq$的形式。\par
剩余的情况是$m$仅有一个素因子,即$m=p^k$。它有因子$1,p,p^2,\cdots,p^k$,它的因子的乘积(除本身)是
\[1\cdot p\cdot p^2\cdot p^3\cdots p^k=p^{(k-1)k/2}\]
这应等于$m=p^k$,所以$(k-1)k/2=k$,得到$k=3$。即如果$m$是积完全数,且只有一个素因子,则它必定形如$p^3$。
%
\exercise a)见code/tools/prime.py
\begin{lstlisting}
def sigma(n):
    factors = factoringPrimeFactors(n)
    res = 1
    for factor in factors:
        res *= factor[0] ** (factor[1] + 1) - 1
        res //= factor[0] - 1
    return res
\end{lstlisting}
b)根据code/exe\_15\_7.py的计算结果,$2\le n\le100$有2个完全数,75个不足数,22个过剩数。$100<n\le200$没有完全数,有76个不足数,24个过剩数。\par
据此可以猜测不足数更普遍。
%
\exercise a)\proof 根据定义$\sigma(n)-n$是$n$是真因子之和。亲和数定义$\sigma(n)-n=m$,$\sigma(m)-m=n$,$\Leftrightarrow$,$\sigma(n)=\sigma(m)=n+m$。\par
b)略\par
c)\proof 
\begin{align*}
\sigma(m)=\sigma(2^epq)&=(2^{e+1}-1)(p+1)(q+1) \\
&=(2^{e+1}-1)\cdot3\cdot2^{e-1}\cdot3\cdot2^e \\
&=3^2\cdot2^{2e-1}\cdot(2^{e+1}-1) \\
\sigma(n)=\sigma(2^er)&=(2^{e+1}-1)(r+1) \\
&=9\cdot2^{2e-1}\cdot(2^{e+1}-1) \\
m+n&=2^epq+2^er \\
&=2^e\left((3\cdot2^{e-1}-1)(3\cdot2^e-1)+(9\cdot2^{2e-1}-1)\right)\\
&=2^e(9\cdot2^{2e-1}-3\cdot2^{e-1}-3\cdot2^e+9\cdot2^{2e-1}) \\
&=2^e(9\cdot2^{2e}-3^2\cdot2^{e-1}) \\
&=2^{2e-1}\cdot9\cdot(2^{e-1}-1)\\
\end{align*}
d)code/exe\_15\_8.py尝试了$2\le e\le 100$的所有情况,并建表如下:
\begin{center}
\begin{tabular}{c|c|c|c|c|c}
\hline
$e$ & $p$ & $q$ & $r$ & $m=2^epq$ & $n=2^er$ \\
\hline
2 & 5 & 11 & 71 & 220 & 284 \\
\hline
4 & 23 & 47 & 1151 & 17296 & 18416 \\
\hline
7 & 191 & 383 & 73727 & 9363584 & 9437056 \\
\hline
\end{tabular}
\end{center}
其他的可能值因为值过大,运行了很久也没能得到,事实上可能超出了整型的范围。
%
\exercise 见Henri Cohen, \emph{Mathematics of Computation} \bf{24} (1970), 423--429\par
a)12496,14288,15472,14536,14264\par
b)1264460,1547860,1727636,1305184\par
c)805984760,1268997640,1803863720,2308845400,3059220620,3367978564,2525983930,2301481286,1611969514\par
d)90632826380,101889891700,127527369100,159713440756,129092518924,106246338676
