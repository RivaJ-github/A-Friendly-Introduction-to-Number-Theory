\chapter{高次幂之和和费马大定理}
\exercise 略
%
\exercise a)根据提示,将$a=xz$,$b=yz$,$c=z^2$代入方程(*)。
\begin{align*}
a^3+b^3=&c^2 \\
x^3z^3+y^3z^3=&z^4 \\
x^3+y^3=&z 
\end{align*}
我们可以任取$x$,$y$,再令$z=x^3+y^3$,就可以得到方程(*)的一个解:
\[(x(x^3+y^3), y(x^3+y^3),(x^3+y^3)^2)\]
运行code/exe\_4\_2\_a.py得到$4\ge y\ge x\ge 1$的解:
\begin{gather*}
(2, 2, 4)
(9, 18, 81)
(28, 84, 784) \\
(65, 260, 4225)
(32, 32, 256)
(70, 105, 1225) \\
(144, 288, 5184)
(162, 162, 2916)
(273, 364, 8281)
\end{gather*}\par
b)略(too simple)\par
c)a)给出的解中,恰好有4个数本源的
\[(2, 2, 4)\, (65, 260, 4225)\, (70, 105, 1225)\, (273, 364, 8281)\]
一般地,如果$x$,$y$互素,且$x^3+y^3$不是某个平方数的倍数,则a)得到的解就是本原的\par
d)$a=b$意味着$2a^3=c^2$,因此$c$是偶数,记$c=2c_1$。则$a^3=2c_1^2$。
则$a$是偶数,记$a=2a_1$。则$4a_1^3=c_1^2$。
再次得到$c_1$是偶数。记$c=2c_1=4c_2$,则$a_1^3=c_2^2$。
一个数既是立方数,又是平方数的唯一可能是它是一个6次方数。故可令$a_1^3=c_2^2=n^6$,
即$a_1=n^2$,$c_2=n^3$。\par
综上,$a=2n^2$,$c=4n^3$。故任何$a=b$的解都形如$(2n^2, 2n^2, 4n^3)$。故$(2,2,4)$是唯一满足$a=b$的本原解。\par
e)考虑c)的结论,我们通过令$x$,$y$互素,$x^3+y^3$不是某个平方数的倍数,来得到本原解。
令$x=1$,则$a=1+y^3$,$b=y(1+y^3)$,$c=(1+y^3)^2$。令$y\ge 22$,使得$a>10000$,并验证$1+y^3$没有平方数因子。
运行code/exe4\_2\_5\_e.py,得到$y=22$,$y=25$,$y=28$时满足要求,本原解分别如下:
\[(10649, 234278, 113401201)\quad
(15626, 390650, 244171876)\quad
(21953, 614684, 481934209)\]
