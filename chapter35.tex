\chapter{数论与虚数}
\begin{theorem}[代数学基本定理]
如果$a_0$,$a_1$,$a_2$,$\cdots$,$a_d$都是复数且$a_0\not=0$,$d\ge1$,则方程
\[a_0x^{d}+a_1x^{d-1}+a_2x^{d-2}+\cdots+a_{d-1}x + a_d=0\]
有复数解。
\end{theorem}
\begin{theorem}[高斯单位定理]
高斯整数仅有的单位是1,$-1$,$i$,$-i$,即1,$-1$,$i$,$-i$是仅有的四个具有高斯整数乘法逆元的高斯整数。
\end{theorem}
高斯整数$\alpha$成为\emph{高斯素数},指能整除$\alpha$的高斯整数只有以下8个数:
\[1,-1,i,-i,\alpha,-\alpha,i\alpha,-i\alpha\]
即整除$\alpha$的只有单位与单位的$\alpha$倍。\par
$x+yi$的范数定义为
\[N(x+yi)=x^2+y^2\]
\begin{theorem}[范数的积性]
设$\alpha$,$\beta$是任意的复数,则
\[N(\alpha\beta)=N(\alpha)N(\beta)\]
\end{theorem}
\begin{theorem}[高斯素数定理]
高斯素数可以描述如下:
\begin{enumerate}
\renewcommand{\labelenumi}{(\roman{enumi})}
\item $1+i$为高斯素数;
\item 设$p$是普通素数,且$p\equiv3\pmod4$,则$p$是高斯素数。
\item 设$p$是普通素数,且$p\equiv1\pmod4$,将$p$表成两平方数之和$p=u^2+v^2$,则$u+vi$是高斯素数。
\end{enumerate}
每个高斯素数等于一个单位($\pm1$或$\pm i$)乘以形式(\romannumeral1)、(\romannumeral2)、(\romannumeral3)中的一个高斯素数。
\end{theorem}
\begin{lemma}[高斯整除性引理]
设$\alpha=a+bi$为高斯整数。
\begin{enumerate}
\renewcommand{\labelenumi}{(\alph{enumi})}
\item 如果2整除$N(\alpha)$,则$1+i$整除$\alpha$;
\item 设$\pi=p$为类别(\romannumeral2)的素数,并假设$p$作为普通整数整除$N(\alpha)$,则$\pi$作为高斯整数整除$\alpha$。
\item 设$\pi=u+vi$为类别(\romannumeral3)中的高斯素数,且$\overline\pi=u-vi$。假定$N(\pi)=p$作为普通整数整除$N(\alpha)$,则$\pi$和$\overline\pi$中至少有一个作为高斯整数整除$\alpha$。
\end{enumerate}
\end{lemma}
%
\exercise 略
%
\exercise 略
%
\exercise a)$11+10i$;b)$-\frac{5}{17}-\frac{14}{17}i$;c)$i$
%
\exercise a)令$x=u+vi$,则$x^2=(u+vi)^2=(u^2-v^2)+2uvi=95-168i$,于是$u^2-v^2=95, uv=-84$,解得$u=12, v=-7$,于是$x=12-7i$\par
b)类似地,令$x=u+vi$,则$x^2=(u+vi)^2=(u^2-v^2)+2uvi=1+2i$,于是$u^2-v^2=1, uv=1$,将$u=1/v$代入前式得$u^4-u^2-1=0$,解得$u^2=(1\pm\sqrt5)/2$,因为$u$是实数,所以舍去负数解。最终得到$v^2=u^2-1=(-1+\sqrt5)/2$,同样舍去负数解。最终得到$1+2i$的两个平方根:
\[\pm\left(\sqrt{\frac{1+\sqrt5}{2}} + \sqrt{\frac{-1+\sqrt5}{2}}i\right)\]
%
\exercise a)不能,$\alpha/\beta=(11-8i)/(3+5i)=(11-8i)(3-5i)/34=(-7+79i)/34$\par
b)能,$\alpha/\beta=(4+7i)/(2-3i)=(4+7i)(2+3i)/13=(-13+26i)/65=-1+2i$\par
c)不能,$\alpha/\beta=(3-5i)/(3-39i)=(3-5i)(3+39i)/1530=(204-102i)/1530$\par
d)能,$\alpha/\beta=(3-39i)/(3-5i)=(3-39i)(3+5i)/34=(204-102i)/34=6-3i$
%
\exercise a)\proof $a+bi$整除$c+di$当且仅当商
\[\frac{c+di}{a+bi}=\frac{(ac+bd) + (ad-bc)i}{a^2+b^2}\]
是高斯整数,而这当且仅当$ac+bd$以及$ab-bc$被$a^2+b^2$整除。\par
b)\proof 记$\alpha=a+bi$,$\beta=c+di$,假设$\alpha$整除$\beta$,可设$\beta=\alpha\gamma$,其中$\gamma$是高斯整数。根据范数积性,$N(\beta)=N(\alpha\gamma)=N(\alpha)N(\gamma)$,所以$N(\alpha)$整除$N(\beta)$,即$a^2+b^2$整除$c^2+d^2$。
%
\exercise 基本运算,略。
%
\exercise a)如果$\alpha=a+bi\sqrt2$是一个单位,则存在$\beta\in R_1$使得$\alpha\beta=1$。两边取范数可得$N(\alpha)N(\beta)=1$,由$N(\alpha)=a^2+2b^2$是整数且非负,可知$N(\alpha)=1$,于是唯一的可能性是$a^2=1$且$b^2=0$。因此$R_1$的单位只有$\pm1$。\par
b)设$\alpha=a+b\rho$,计算范数
\begin{align*}
N(\alpha)&=(a+b\rho)(a+b\bar\rho) \\
&=a^2+b^2\rho\bar\rho + ab(\rho+\bar\rho) \\
&=a^2+b^2-ab
\end{align*}
与a)相同,假设$\alpha$是单位,则$a^2-ab+b^2=1$,两边乘4并配方:
\[(2a-b)^2+3b^2=4\]
显然$b^2\le1$。如果$b=0$,则$a=\pm1$。如果$b=1$,则$2a-b=2a-1=\pm1$,所以$a=0$或$1$。最后,如果$b=-1$,则$2a-b=2a+1=\pm1$,得到$a=0$或$-1$。综上,$R_2$的单位有六个:$\pm1$,$\pm\rho$,$\pm(1+\rho)$。换种方式描述是$\{1,\rho,\rho^2,\rho^3,\rho^4,\rho^5\}$。\par
c)$R_3$的单位是$a/d$,其中$p\nmid a$且$p\nmid b$
%
\exercise a)\proof 设$\alpha=a+b\sqrt3$,$\beta=c+d\sqrt3$,则$\alpha\beta=(a+b\sqrt3)(c+d\sqrt3)=(ac+3bd)+(ad+bc)\sqrt3$,于是
\begin{align*}
N(\alpha\beta)&=(ac+3bd)^2-3(ad+bc)^2 \\
&=a^2c^2+9b^2d^2-3a^2d^2-3b^2c^2 \\
&=(a^2-3b^2)(c^2-3d^2) \\
&=N(\alpha)N(\beta)
\end{align*}
证毕。\par
b)\proof 如果$\alpha$是单位,则存在$\beta$使得$N(\alpha)N(\beta)=1$,由于$N(\alpha)=a^2-3b^2$是整数,所以$N(\alpha)=\pm1$。假设$N(\alpha)=a^2-3b^2=-1$,两边模3可得$a^2\equiv-1\pmod3$,但由二次互反律,上式无解,所以$N(\alpha)=1$,证毕。\par
c)已知$N(\alpha)=a^2-3b^2=1$,存在$\beta=a-b\sqrt3$,使得$\alpha\beta=(a+b\sqrt3)(a-b\sqrt3)=a^2-3b^2=1$,所以$\alpha$是$R$的单位。\par
d,e)即求方程$a^2-3b^2=1$的解。由最小的正整数解是$(2,1)$,记$u+v\sqrt3=(2+\sqrt3)^k$,其中$k$可以是任意整数,则$\pm u\pm v\sqrt3$是$R$的单位。
%
\exercise 略
%
\exercise a)首先分解范数$N(91+63i)=12250=2\cdot5^3\cdot7^2$。根据引理,$1+i$和7整除$91+63i$,所以我们计算$(91+63i)/7=13+9i$,$(13+9i)/(1+i)=11-2i$。接着由于$5=2^2+1^2$,由引理可知它被$2+i$或$2-i$整除。由于$(11-2i)/(2-i)=(24+7i)/5$,所以正确的是$2+i$,计算$(11-2i)/(2-i)=4-3i$。由$N(4-3i)=25=5^2$,所以还能被$2+i$或$2-i$整除,$(4-3i)/(2-i)=1-2i$,而$1-2i$是和$2+i$相同高斯素数,$-i(2+i)=1-2i$。综上得到最终分解
\[91+63i=-i\cdot(1+i)\cdot(2+i)^3\cdot7\]
b)$975=3\cdot5^2\cdot13$。3是高斯素数,5和13很容易分解为高斯素数:$5=(2+i)(2-i)$,$13=(3+2i)(3-2i)$。于是$975=3\cdot(2+i)^2\cdot(2-i)^2\cdot(3+2i)^2\cdot(3-2i)^2$\par
c)$N(53+62i)=6653$是普通素数,所以$53+62i$是(\romannumeral3)类高斯素数

