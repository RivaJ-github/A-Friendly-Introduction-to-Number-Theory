\chapter{哪些素数可表成两个平方数之和}
\begin{theorem}[素数的两个平方数之和定理]
设$p$为奇素数,则$p$是两个平方数之和的充要条件是
\[p\equiv1\pmod4\quad\text{(或$p=2$)}\]
\end{theorem}
%
\exercise a)前几个能表示成$a^2+ab+b^2$的素数是:3,7,13,19,31,37,43,61,67,73,79,97,103,109。除了3,其他都是模3余1的数(由于他们都是奇数,所以也是模6余1的。下面给出必要性证明:\par
考虑$a$和$b$模3的所有可能性。
\begin{center}
\begin{tabular}{lll}
$a\equiv0\pmod3$ & $b\equiv0\pmod3$ & $a^2+ab+b^2\equiv0\pmod3$ \\
$a\equiv0\pmod3$ & $b\equiv1\pmod3$ & $a^2+ab+b^2\equiv1\pmod3$ \\
$a\equiv0\pmod3$ & $b\equiv2\pmod3$ & $a^2+ab+b^2\equiv1\pmod3$ \\
$a\equiv1\pmod3$ & $b\equiv1\pmod3$ & $a^2+ab+b^2\equiv0\pmod3$ \\
$a\equiv1\pmod3$ & $b\equiv2\pmod3$ & $a^2+ab+b^2\equiv1\pmod3$ \\
$a\equiv2\pmod3$ & $b\equiv2\pmod3$ & $a^2+ab+b^2\equiv0\pmod3$ \\
\end{tabular}
\end{center}
因此$a^2+ab+b^2$模3余1或0。但如果它是一个素数,则它不能被3整除,或它等于3。因此,如果$p=a^2+ab+b^2$是不同于3的素数,则$p\equiv1\pmod3$\par
b)前几个能表示成$a^2+2b^2$的素数是:2,3,11,17,19,41,43,59,67,73,83,89,97,119。他们具有性质$p\equiv1$或$3\pmod8$(除了2)。\par
\proof 如果$p=a^2+2b^2$,则$a$必须是奇数,不妨设$a=2k+1$。则
\begin{align*}
p&=(2k+1)^2+2b^2\\
&=4k^2+4k+1+2b^2\\
&=4k(4k+1)+1+2b^2\\
&\equiv1+2b^2\pmod8
\end{align*}
如果$b$是偶数,则$2b^2\equiv0\pmod8$,于是$p\equiv1\pmod8$。如果$b$是奇数,不妨记$b=2m+1$,则
\begin{align*}
1+2b^2&=1+2(2m+1)^2\\
&=1+2(4m^2+4m+1)\\
&\equiv3\pmod8
\end{align*}
所以$p\equiv3\pmod8$。
%
\exercise 对$p$模5得,$p\equiv a^2\pmod5$,这意味着除非$p=5$,否则$p$是模5的二次剩余,所以$p\equiv1,4\pmod5$。对$p$模4得,$p\equiv a^2+b^2\pmod4$,$a$和$b$奇偶性不同,否则$p$就是偶数。不妨设$a=2m+1$,$b=2n$,则$p\equiv (2k+1)^2+(2m)^2\equiv1\pmod4$。所以$p\equiv1,4\pmod5$且$p\equiv1\pmod4$,唯一可能的是$p\equiv1,9\pmod {20}$
%
\exercise 一次递降后得到$242^2+41^2=5·12049$,二次递降后得到$105^2+32^2=12049$
%
\exercise a)$23^2+38^2=1973$;b)$258^2+173^2=96493$
%
\exercise a)运行code/exe\_24\_5.py得到如下结果:
\begin{align*} 
    &2=1^2+1^2+0^2\\
    &3=1^2+1^2+1^2\\
    &5=2^2+1^2+0^2\\
    &7=None\\
    &11=3^2+1^2+1^2\\
    &13=3^2+2^2+0^2\\
    &17=3^2+2^2+2^2=4^2+1^2+0^2\\
    &19=3^2+3^2+1^2\\
    &23=None\\
    &29=4^2+3^2+2^2=5^2+2^2+0^2\\
    &31=None\\
    &37=6^2+1^2+0^2\\
    &41=4^2+4^2+3^2=5^2+4^2+0^2=6^2+2^2+1^2\\
    &43=5^2+3^2+3^2\\
    &47=None\\
    &53=6^2+4^2+1^2=7^2+2^2+0^2\\
    &59=5^2+5^2+3^2=7^2+3^2+1^2\\
    &61=6^2+4^2+3^2=6^2+5^2+0^2\\
    &67=7^2+3^2+3^2\\
    &71=None\\
    &73=6^2+6^2+1^2=8^2+3^2+0^2\\
    &79=None\\
    &83=7^2+5^2+3^2=9^2+1^2+1^2\\
    &89=7^2+6^2+2^2=8^2+4^2+3^2=8^2+5^2+0^2=9^2+2^2+2^2\\
    &97=6^2+6^2+5^2=9^2+4^2+0^2
\end{align*}
b)在小于100的素数中,只有7,23,31,47,71和79不能被分解为三个平方数的和。不难发现他们都是模8余7的数。因此可以给出下列结论(排除2):
\begin{enumerate}
    \item 如果$p\equiv 1,3,5\pmod8$,则$p$是三个平方数的和。
    \item 如果$p\equiv 7\pmod8$,则$p$不可能是三个平方数的和。
\end{enumerate}
c)证明结论2:如果$a$是偶数,则$a^2\equiv 1\pmod8$;如果$a$是偶数,则$a^2\equiv 0,4\pmod8$。如果将三个平方数的和模8,则有以下9种组合:
\begin{align*}
0+0+0\equiv0 && 0+0+1\equiv1 && 0+0+4\equiv4 && 0+1+1\equiv2 && 0+1+4\equiv5 \\
0+4+4\equiv0 && 1+1+4\equiv6 && 1+4+4\equiv1 && 4+4+4\equiv4 &&  
\end{align*}
因此,不存在三个平方数的和模8余7的情况。
%
\exercise a)在递降程序的证明过程中,仅第(\romannumeral1)步用到了$p$是模4余1素数这一条件,是为了说明$x^2\equiv-1\pmod p$有解,可写成$x^2+1^2=Mp$的形式。\par
b)$7**2 + 4**2 =65$\par
c)递降程序需要找到一个形式$A^2+B^2=Mc$的初始形式,如果$-1$是模$c$的二次剩余,则可以轻松地得到$A^2+1^2=Mc$。但一般而言,$c\equiv-1\pmod 4$的假设并不意味着$-1$是模$c$的二次剩余。例如,$-1$不是模21的二次剩余,21也不是两个平方数的和。
%
\exercise a)
\begin{lstlisting}
def DesentProcedure(A, B, p):
    ''' 
    费马二次递降程序
    start from A^2 + B^2 \equiv (mod p)
    @return (u, v) 满足a^2 + b^2 = p
    '''
    M = (A**2 + B**2) // p
    while (M > 1):
        u = A % M
        if (u > M // 2):
            u -= M
        v = B % M
        if (v > M // 2):
            v -= M
        r = (u**2 + v**2) // M
        A, B = abs(A*u + B*v) // M, abs(A*v - B*u) // M
        M = r
    return (A, B)
\end{lstlisting}
b)
\begin{lstlisting}
def DesentProcedure_1(p):
    ''' 
    对满足p模8余5的p进行二次递降程序
    @return (u, v) 满足a^2 + b^2 = p
    '''
    A = (-2 * successive_square(-4, (p-5)//8, p)) % p
    return DesentProcedure(A, 1, p)
\end{lstlisting}
