\chapter{丢番图逼近与佩尔方程}
\exercise a)建表如下,但无法找到一定的模式(事实上没有人做到):
\begin{center}
\begin{tabular}{c|c}
$D$ & $x^2-Dy^2=-1$的解 \\
\hline
2 & (1,1) \\
3 & None \\
5 & (2,1) \\
6 & None \\
7 & None \\
8 & None \\
10 & (3,1) \\
11 & None \\
12 & None \\
13 & (18,5) \\
14 & None \\
15 & None \\
17 & (5,1) \\
18 & None \\
19 & None \\
20 & None \\
\end{tabular}
\end{center}
b)
\begin{align*}
&(x_0^2+Dy_0^2)^2-D(2x_0y_0)^2 \\
=&x_0^4-2Dx_0^2y_0^2 + D^2y_0^4 \\
=&(x_0^2-Dy_0^2)^2=(-1)^2=1
\end{align*}
c)当$y=5$,$41y^2-1=1024=32^2$,所以$(32,5)$是$x^2-41y^2=-1$的一个解。根据b)$(32^2+41\cdot5^2,2\cdot32\cdot5)=(2049,320)$是佩尔方程$x^2-41y^2=1$的解。\par
d)
\begin{align*}
&(x_0x_1+Dy_0y_1)^2 - D(x_0y_1+y_0x_1)^2 \\
=&x_0^2x_1^2+D^2y_0^2y_1^2 -Dx_0^2y_1^2-Dy_0^2x_1^2 \\
=&(x_0^2-Dy_0^2)(x_1^2-Dy_1^2) \\
=&1\cdot M=M
\end{align*}
首先找到$x^2-2y^2=7$的一个解$(5,3)$,再找到$x^2-2y^2=1$的一个解$(3,2)$。然后通过上式进行迭代,得到以下解:
\begin{gather*}
(27, 19),\quad (157, 111),\quad (915, 647),\quad (5333, 3771),\quad (31083, 21979) \\
\quad (181165, 128103),\quad (1055907, 746639),\quad (6154277, 4351731)
\end{gather*}
%
\exercise a)如果有解,则$x^2\equiv 7\pmod{11}$,但是7不是模11的二次剩余($\left(\frac{7}{11}\right)=-\left(\frac{11}{7}\right)=-\left(\frac{4}{7}\right)=-1$),所以方程无解。\par
b)$x^2\equiv4\pmod{11}$,所以$x\equiv\pm2\pmod{11}$,另外,由$x>\sqrt{433}\approx20.8$,所以我们可以尝试$x=24,31,35,42,46,\cdots$,并检查$(x^2-433)/11$是否完全平方数。最终发现$x=42$时,有$(42^2-433)/11=121=11^2$,所以我们找到一个解$(42,31)$。这是最小的解。接下来的三个解分别是$(57,16),(783,236),(1098,331)$。\par
c)模3可得$x^2-2y^2\equiv0\pmod3$,即$x^2\equiv2y^2\pmod3$。由于2是3的非二次剩余,所以必须有$x\equiv y\equiv 0\pmod{3}$。因此可设$x=3X$,$y=3Y$,方程变为$9X^2-99Y^2=3$,这意味着9要整除3,产生矛盾,所以$x^2-11y^2=3$无解。
