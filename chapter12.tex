\chapter{素数}
\begin{theorem}[无穷多素数定理]
存在无穷多个素数
\end{theorem}
\begin{theorem}[模4余3的素数定理]
存在无穷多个模4余3的素数
\end{theorem}
\begin{theorem}[算术级数的素数狄利克雷定理]
设$a$与$m$是整数,$\gcd(a,m)=1$,则存在无穷多个素数$p$满足
\[p\equiv a\pmod m\]
\end{theorem}
%
\exercise 见下表:
\begin{align*}
&\{5\} &&A=5+1 && =6 &&=2\cdot3 \\
&\{5,2\} &&A=5\cdot2+1 &&=11 && \\
&\{5,2,11\} &&A=5\cdot2\cdot11+1 && =111 &&=3\cdot37 \\
&\{5,2,11,3\} &&A=5\cdot2\cdot11\cdot3+1 && =331 && \\
&\{5,2,11,3,331\} &&A=5\cdot2\cdot11\cdot3\cdot31+1 && =109231 &&
\end{align*}
\exercise a)令$p_1,\cdots p_r$是任意给定的模6余5的素数集合。假设5不在集合中。令$A=6p_1p_2\cdots p_r+5$,并假设分解$A=q_1q_2\cdots q_s$。因为$A\equiv 5\pmod6$,$q_i$不可能是2或3。这意味着$q_i$同余1或5$\pmod6$。但如果所有$q_i$都同余1$\pmod6$,则$A$也同余1$\pmod6$,所以这不可能,即至少存在一个$q_i$模6余5,不妨将其记为$q$。如果$q$在原集合中,则$q$整除$A$和$p_1p_2\cdots p_r$,也将整除它们的差5。我们已经5排除在原集合外。所以我们得到一个新的素数$q\equiv5\pmod6$。我们可以重复这个过程得到无穷多个模6余5的素数。\par
b)问题在于如果$q_1q_2\cdots q_r$模5余4,不能得出至少存在一个因子是模5余4。例如,两个因子都是模5余2,它们的乘积是模5余4的。另外如果两个因子都是模5余3的,它们的乘积也是模5余4的。举个实际的例子,我们从$\{19\}$开始,$A=5\cdot19+1=99=3^2\cdot11$。$A$有两个素因子3和11,它们都不是模5余4的。
%
\exercise a)将和式分组计算:
\[\frac{1}{k}+\frac{1}{p-k}=\frac{p}{k(p-k)}\]
由于分组后每一项的分子都有因子$p$,而每一项的分母都没有因子$p$,所以和式的分子$A_p\equiv0\pmod p$。\par
b)当$p\ge5$时,$A_p\equiv0\pmod{p^2}$。沃尔斯滕霍尔姆定理(Wolstenholme's Theorem)
%
\exercise a)如果$\gcd(a, m) = 1$,则$\gcd(a, m-a)=1$。所以如果$m$是奇数,则和式可分组为
\[\frac{1}{a}+\frac{1}{m-a}=\frac{m}{a(m-a)}\]
每一组的和的分子都能被$m$整除,而分母不能被$m$整除。所以和式的分子也被$m$整除。如果$m$是偶数,除非$a=m/2$与$m$互素,否则也能两两分组。而这只发生在$m=2$时。\par
b)很难,略
%
\exercise 下面记$v_p(N)$表示整除$N$的$p$的最高次幂。
a)$v_2(1!)=0$,$v_2(2!)=1$,$v_2(3!)=1$,$v_2(4!)=3$,$v_2(5!)=3$,$v_2(6!)=4$,$v_2(7!)=4$,$v_2(8!)=7$,$v_2(9!)=7$,$v_2(10!)=8$,$v_2(11!)=8$,$v_2(12)=10$,$v_2(13)=10$,$v_2(14)=11$,$v_2(15)=11$,$v_2(16)=15$,$v_2(17)=15$,$v_2(18)=16$,$v_2(19)=16$,$v_2(20)=18$\par
b,c)1到$n$之间的偶数提供一个因子2,4的倍数额外再提供一个2,8的倍数额外再提供一个2,以此类推。
\[v_2(n!)=[n/2]+[n/4]+[n/8]+\cdots=\sum_{k=1}^\infty [n/2^k]\]
其中$[x]$表示不大于$x$的最大整数。\par
d,e)同样的分析方式,可得到
\[v_p(n!)=[n/p]+[n/p^2]+[n/p^3]+\cdots=\sum_{k=1}^\infty [n/p^k]\]
f)由于$[x]\le x$,所以
\[v_p(n!)<n/p+n/p^2+n/p^3+\cdots\]
上式是严格小于的,因为实际的$v_p(n!)$仅有有限项。应用等比数列公式。
\[m\le v_p(n!)<\frac{n}{p-1}\]
%
\exercise a)因为$\gcd(1338, 1115)=223$,而$223$是素数,所以$p=223$满足$p\equiv 1338\pmod{1115}$。且这是唯一满足要求的素数解,因为$x\equiv 1338\pmod{1115}$的每一个解都应整除233。\par
b)由于$\gcd(1438,1115)=1$,根据算术级数的素数狄利克雷定理,存在无穷个素数满足$p\equiv 1438\pmod{1115}$。