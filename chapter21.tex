\chapter{-1是模p平方剩余吗?2呢}
\begin{theorem}[欧拉准则]
设$p$为一个奇素数,则
\[a^{(p-1)/2}\equiv\left(\frac{a}{p}\right)\pmod p\]
\end{theorem}
\begin{theorem}[二次互反律---第\romannumeral1部分]
设$p$为奇素数,则
\begin{gather*}
p\equiv1\pmod4\text{时$-1$是模$p$的二次剩余} \\
p\equiv3\pmod4\text{时$-1$是模$p$的二次非剩余}
\end{gather*}
换句话说,用勒让德符号可以表示为
\[\left(\frac{-1}{p}\right)=
\begin{cases}
    1&\text{当$p\equiv1\pmod4$时}\\
    -1&\text{当$p\equiv3\pmod4$时}\\
\end{cases}\]
\end{theorem}
\begin{theorem}[模4余1素数定理]
存在无穷多个素数与1模4同余。
\end{theorem}
\begin{theorem}[二次互反律---第\romannumeral2部分]
设$p$为奇素数,则当$p$模8余1或7时,2是模$p$当二次剩余;当$p$模8余3或5时,2是模$p$的二次非剩余。用勒让德符号可以表示为
\[\left(\frac{2}{p}\right)=
\begin{cases}
    1&\text{当$p\equiv1$或$7\pmod8$时}\\
    -1&\text{当$p\equiv3$或$5\pmod8$时}\\
\end{cases}\]
\end{theorem}
%
\exercise a)$5987\equiv3\pmod4$,所以无解。\par
b)注意到$6780\equiv-1\pmod{6781}$,且$6781\equiv1\pmod4$,所以有解(解为$x\equiv995$和$x\equiv5786$)。\par
c)配方$x^2+14x-35=(x+7)^2-84$,所以$(x+7)^2\equiv 84\pmod{337}\Rightarrow (2x+14)^2\equiv336\equiv-1\pmod{337}$。因为$337\equiv1\pmod4$,所以同余式有解。\par
事实上$148^2\equiv-1\pmod{337}$,所以$x\equiv67\pmod{337}$是原式的解。$189^2\equiv(189+337)^2\equiv-1\pmod{337}$,所以$x\equiv256\pmod{337}$是原式的解。\par
d)$x^2-64x+943=(x-23)(x-47)$,所以$x\equiv23\pmod{3011}$和$x\equiv47\pmod{3011}$就是原式的解。
%
\exercise 
\begin{align*}
A&=(2p_1)^2+1=1157=13\cdot89,\quad \text{所以}p_2=13 \\
A&=(2p_1p_2)^2+1=1157=5\cdot41\cdot953,\quad \text{所以}p_3=5 \\
A&=(2p_1p_2p_3)^2+1=4884101,\quad \text{所以}p_4=4884101
\end{align*}
%
\exercise 模12时发现规律:
\begin{align*}
&11, 13, 23, 37, 47, 59, 61, 71, 73, 83, 97, 107, 109\\
\equiv&11, 1, 11, 1, 11, 11, 1, 11, 1, 11, 1, 11, 1\pmod{12}\\
&5, 7, 17, 19, 29, 31, 41, 43, 53, 67, 79, 89, 101, 103, 113, 127\\
\equiv & 5, 7, 5, 7, 5, 7, 5, 7, 5, 7, 7, 5, 5, 7, 5, 7\pmod{12}
\end{align*}
有理由给出猜想
\[\left(\frac{3}{p}\right)=
\begin{cases}
    1&\text{当$p\equiv1$或$11\pmod{12}$时}\\
    -1&\text{当$p\equiv5$或$7\pmod{12}$时}\\
\end{cases}\]
%
\exercise 略,数个数的事
%
\exercise a)设$p=10k+1$,(因为考虑奇素数,所以不必考虑$5k'+1$的形式)则$(p-1)/2=5k$,所以我们要将$5,10,15,\cdots,25k$模$p$简化到$-5k$到$5k$,并找出其中有几个负数。
\begin{align*}
&5,10,\cdots,5k: & k\text{个正数} \\
&5k+5,5k+10,\cdots,10k: & k\text{个负数} \\
&10k+5,10k+10,\cdots,15k: & k\text{个正数} \\
&15k+5,15k+10,\cdots,20k: & k\text{个负数} \\
&20k+5,20k+10,\cdots,25k: & k\text{个正数} \\
\end{align*}
总计$2k$个负数,所以
\[5^{(p-1)/2}\equiv(-1)^{2k}\equiv1\pmod p\]
由欧拉准则可知5是模$p$的二次剩余。\par
b)类似地,设$p=10k+7$。则$(p-1)/2=5k+3$,这次要简化$5,10,15,\cdots,25k+15$,模$p$时
\begin{align*}
&5,10,\cdots,5k: & k\text{个正数} \\
&5k+5,5k+10,\cdots,10k+5: & k+1\text{个负数} \\
&10k+10,10k+15,\cdots,15k+10: & k+1\text{个正数} \\
&15k+15,15k+20,\cdots,20k+10: & k\text{个负数} \\
&20k+15,20k+20,\cdots,25k+15: & k+1\text{个正数} \\
\end{align*}
总计$2k+1$个负数,所以
\[5^{(p-1)/2}\equiv(-1)^{2k+1}\equiv-1\pmod p\]
由欧拉准则可知5是模$p$的二次非剩余。
%
\exercise 我们的猜想是:
\[p\equiv1\pmod4\Longrightarrow A=B\]
\proof 已知
\[p\equiv1\pmod4\Longrightarrow -1\text{是模$p$的二次剩余}\]
设$n_1,\cdots n_r$是1到$(p-1)/2$之间二次剩余。由于$-1$也是二次剩余,所以$-n_1,\cdots,-n_r$也是二次剩余。这意味着$p-n_1,\cdots,p-n_r$也是二次剩余。所以1到$p-1$的二次剩余是
\[n_1,n_2,\cdots,n_r,p-n_1,p-n_2,\cdots,p-n_r\]
对上式求和,得到$A=pr$。二次剩余的总个数是$2r$,而我们知道$2r=(p-1)/2$,所以$r=(p-1)/4$,$A=(p^2-p)/4$。又$A+B=\sum_{i=1}^{p-1}i=(p^2-p)/2$。所以$A=B=(p^2-p)/4$,证毕。