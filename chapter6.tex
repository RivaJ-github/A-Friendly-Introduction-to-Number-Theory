\chapter{线性方程和最大公因数}
\begin{theorem}[线性方程定理]
设$a$与$b$是非零整数,$g=\gcd(a,b)$。方程
\[ax+by=g\]
总是有一个整数解$(x_1, y_1)$,它可由欧几里得算法得到。则方程的每一个解可由
\[\left(x_1+k\cdot\frac{b}{g}, y_1-k\cdot\frac{a}{g}\right)\]
得到,其中$k$是任意整数。
\end{theorem}
%
\exercise a)$12345\cdot 11 - 67890\cdot 2=\gcd(12345, 67890) = 15$\par
b)$-54321\cdot 1645 + 9876\cdot 9048 = \gcd(54321, 9876) = 3$
%
\exercise 
a)$105\cdot(-53)+121\cdot46=1$。一般解为$(-53+121k, 46-105k)$\par
b)$12345\cdot11+67890\cdot(-2)=15$。一般解为$(11+4526k, -2-823k)$\par
c)$54321\cdot(-1645)+9876\cdot9048=3$。一般解为$(-1645+3292k, 9048-18107k)$
%
\exercise 
a){\bf 提示:}每次迭代后$w$是新产生的余数,$g$上一次迭代产生的余数。$v$是形如$r_i = ax + by$中的$x$。而$x$则记录的是$r_{i-1} = ax + by$中的$x$。\par
b)python实现如下:
\begin{lstlisting}
def gcdWithXY(a, b):
    x, g, v, w = 1, a, 0, b
    while (w != 0):
        t = g % w
        q = g // w
        s = x - q * v
        x, g = v, w
        v, w = s, t
    return (g, x, (g - a * x) // b)
\end{lstlisting}
c)运行code/exe6\_3.py。
\begin{gather*}
\gcd(19789, 23548) = 7\quad (x,y)=(1303,-1095) \\
\gcd(31875, 8387) = 1\quad (x,y)=(-381,1448) \\
\gcd(22241739, 19848039) = 237\quad (x,y)=(-8980,10063)
\end{gather*}
d)添加特殊处理:
\begin{lstlisting}
    if (b == 0):
        return (a, 1, 0)
\end{lstlisting}
e)在返回前添加校验:
\begin{lstlisting}
    while (x <= 0):
        x += b
\end{lstlisting}

%
\exercise 