\chapter{模p平方剩余}
\begin{theorem}
设$p$为一个奇素数,则恰有$\frac{p-1}{2}$个模$p$的二次剩余,且恰有$\frac{p-1}{2}$个模$p$的二次非剩余。
\end{theorem}
\begin{theorem}[二次剩余乘法法则---版本1]
设$p$为奇素数,则
\begin{enumerate}
\item 两个模$p$的二次剩余的积是二次剩余。
\item 二次剩余与二次非剩余的积是二次非剩余。
\item 两个二次非剩余的积是二次剩余。
\end{enumerate}
这三条法则可用符号表示如下:
\[QR\times QR=QR,\quad QR\times NR=NR,\quad NR\times NR=QR\]
\end{theorem}
{\itshape $a$模$p$的勒让德符号是
\[\left(\frac{a}{p}\right)=
\begin{cases}
1 & \text{若$a$是模$p$的二次剩余} \\
-1 & \text{若$a$是模$p$的二次非剩余} \\
\end{cases}
\]}
\begin{theorem}[二次剩余乘法法则---版本2]
设$p$为奇素数,则
\[\left(\frac{a}{p}\right)\left(\frac{b}{p}\right)=\left(\frac{ab}{p}\right)\]
\end{theorem}
%
\exercise 二次剩余:$1, 4, 5, 6, 7, 9, 11, 16, 17$\par
二次非剩余:$2,3,8,10,12,13,14,15,18$
%
\exercise a)
\begin{center}
\begin{tabular}{c|c|c}
$p$ & $A$ & $B$ \\
\hline 
3 & 1 & 2 \\
5 & 5 & 5 \\
7 & 7 & 14 \\
11 & 22 & 33 \\
13 & 39 & 39 \\
17 & 68 & 68 \\
19 & 76 & 95 \\
\end{tabular}
\end{center}
b)$A+B$是1到$p-1$的所有数字和,即$1+2+\cdots+(p-1)=(p-1)p/2$\par
c)观察发现$p\ge5$时,$A\equiv B\equiv0\pmod p$。\par
\proof 利用平方和公式:
\[1^2+2^2+\cdots + N^2=N(N+1)(2N+1)/6\]
令$N=(p-1)/2$,得:
\[A\equiv 1^2+2^2+\cdots + \left(\frac{p-1}{2}\right)^2=\equiv\frac{p^3-p}{24}\pmod p\]
因为$24=2^3\cdot3$,所以$p\ge5$时,24并不能消去分子中的$p$,所以$(p^3-p)/24\equiv 0\pmod p$,证毕。\par
d)100以内的满足$A=B$的素数有$\{5,13,17,29,37,41,53,61,73,89,97\}$,都满足$p\equiv1\pmod4$。有理由猜测:
\[p\equiv1\pmod4 \Longrightarrow A=B\]\par
e)算法如下:
\begin{lstlisting}
def quadratic_residue(p):
    '''计算素数p的二次剩余列表'''
    res = []
    for a in range(1, (p+1)//2):
        res.append(a**2 % p)
    return res

def getAB(p):
    lst = quadratic_residue(p)
    A = 0
    for i in lst:
        A += i
    B = (p - 1) // p * 2 - A
    return (A, B)
\end{lstlisting}
测试结果均满足$B\ge A$,但还没有已知的初等证明方式。
%
\exercise a)
\begin{center}
\begin{tabular}{c|c}
$p$ & 三次剩余 \\
\hline
5 & {1, 2, 3, 4} \\
7 & {1, 6} \\
11 & {1, 2, 3, 4, 5, 6, 7, 8, 9, 10} \\
13 & {8, 1, 12, 5} \\
\end{tabular}
\end{center}\par
b)19的三次剩余有:\{1, 7, 8, 11, 12, 18\}。2和4都不是三次剩余,但$8=2\cdot4$是三次剩余。2和3和$6=2\cdot3$都不是三次剩余\par
c)如果$p\equiv2\pmod3$,则1到$p-1$的所有数都是模$p$的三次剩余。\par
\proof 由于$\gcd(3, p-1)$,所以下面的同余式有解:
\[3u+(p-1)v=1\]
任取$a\not\equiv p$,令$b=a^u$,可得
\[b^3=a^{3u}=a^{1-(p-1)v}\equiv a\pmod p\]
所以$a$是模$p$的三次剩余,证毕。