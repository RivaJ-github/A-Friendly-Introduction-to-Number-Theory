\chapter{哪些数能表成两个平方数之和}
\begin{theorem}[两平方数之和定理]
设$m$是正整数。
\begin{enumerate}
\item 将$m$分解为
\[m=p_1p_2\cdots p_r M^2\]
其中$p_1$,$p_2$,$\cdots$,$p_r$是互不相同的素因子,则$m$可表成两个平方数之和的充要条件是每个$p_i$或为2或为模4余1.
\item $m$能表成两平方数之和$m=a^2+b^2$且$\gcd(a,b)=1$,当且仅当一下两个条件之一成立:
\begin{enumerate}
\item $m$是奇数且$m$的每个素因子都模4余1;
\item $m$是偶数,$m/2$是奇数且$m/2$的每个素因子都模4余1。
\end{enumerate}
\end{enumerate}
\end{theorem}
\begin{theorem}[毕达哥拉斯斜边命题]
$c$是一个本原勾股数组斜边的充要条件是:$c$是模4余1的素数的乘积。
\end{theorem}
%
\exercise a)$4370=2\cdot5\cdot19\cdot23$,因为$19\equiv3\pmod4$,所以它不能表成两个平方数之和。\par
b)
\begin{align*}
1885&=5\cdot13\cdot29 \\
&=(1^2+2^2)(2^2+3^2)(2^2+5^2)\\
&=(8^2+1^2)(2^2+5^2) \\
&=21^2+38^2 
\end{align*}
c)
\begin{align*}
1885&=29\cdot41\\
&=(2^2+5^2)(4^2+5^2)\\
&=33^2+10^2
\end{align*}
d)
\begin{align*}
3185&=5\cdot7^2\cdot13\\
&=(1^2+2^2)(2^2+3^2)7^2\\
&=(8^2+1^2)(7^2)\\
&=7^2+56^2
\end{align*}
%
\exercise a)偶数不能作为本原勾股数组的斜边。\par
b)
\begin{align*}
&s=61\quad t=7\quad (a,b,c)=(427, 1836, 1885)\\
&s=59\quad t=17\quad (a,b,c)=(1003, 1596, 1885)\\
&s=53\quad t=31\quad (a,b,c)=(1643, 924, 1885)\\
&s=49\quad t=37\quad (a,b,c)=(1813, 516, 1885)
\end{align*}
c)
\begin{align*}
&s=47\quad t=13\quad (a,b,c)=(611, 1020, 1189)\\
&s=43\quad t=23\quad (a,b,c)=(989, 660, 1189)
\end{align*}
d)没有本原勾股数组以3185为斜边,但是有两个非本原勾股数组:
\begin{align*}
&s=77\quad t=21\quad (a,b,c)=(1617, 2744, 3185)\\
&s=63\quad t=49\quad (a,b,c)=(3087, 784, 3185)
\end{align*}
%
\exercise $(c-a)(c+a)=c^2-a^2=5929=7^2\cdot11^2$。将5929分解为两数乘积,有四种可能$(c-a, c+a)\in\{(1,5929),(7,847),(11,539),(49,121)\}$,对应四组解:
\[(a,c)\in\{(36,85),(264,275),(420,427),(2964,2965)\}\]
%
\exercise 令$p=p_i\equiv3\pmod4$。假设$m=a^2+b^2$,则$a^2\equiv-b^2\pmod p$。由于$-1$不是模$p$的二次剩余,所以$p\mid a$且$p\mid b$。于是$p^2\mid m$,所以$p\mid M$。所以我们可以得到$m/p^2=p_1p_2\cdots p_r(M/p)^2$且$m/p^2=(a/p)^2+(a/p)^2$。重复此过程,我们会发现$p^2\mid m/p^2$。此过程终将停止,并引发矛盾。
%
\exercise 略。TODO: 有点难
%
\exercise a)运行code/exe\_25\_6.py得到如下结果:\par
(\romannumeral1)$S(10)=1\quad n=3^2+1^2$\par
(\romannumeral2)$S(70)=0$\par
(\romannumeral3)$S(130)=2\quad n=11^2+3^2=9^2+7^2$\par
(\romannumeral4)$S(1105)=4\quad n=33^2+4^2=32^2+9^2=31^2+12^2=24^2+23^2$\par
b、c、d)$S(p_1p_2\cdots p_r)=2^{r-1}$,所以$S(p)=1$,$S(pq)=2$。TODO:证明有点难。
%
\exercise TODO: 经过了几组测试,还不确定是否完全正确
\begin{lstlisting}
def DesentProcedure_2(n):
    '''
    利用二次递降程序分解n
    @return (x, y) 满足x^2 + y^2 = n
    '''
    factors = factoringPrimeFactors(n)
    c = 1 # x, y的公因数
    x, y = 0, 1
    for f in factors:
        (p, r) = f
        c = p ** (r // 2)
        if (r % 2 == 1):
            if p % 4 == 3:
                raise Exception('n cannot be decomposed')
            while True:
                a = random.randint(2, p)
                if JacobiSymbol(a, p) == -1:
                    A = successive_square(a, (p-1)//4, p)
                    if (A != 1): # 不知道为什么会出现1
                        break
                else:
                    print('Can false?')
            u, v = DesentProcedure(A, 1, p)
            x, y = x * u + y * v, abs(x * v - y * u)
    return (x * c, y * c)
\end{lstlisting}