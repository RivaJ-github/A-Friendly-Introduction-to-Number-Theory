\chapter{同余式、幂与费马小定理}
\begin{theorem}[费马小定理]
设$p$是素数,$a$是任意整数且$a\not\equiv0\pmod p$,则
\[a^{p-1}\equiv 1\pmod p\]
\end{theorem}
\begin{lemma}
设$p$为素数,$a$是任意整数且$a\not\equiv0\pmod p$,则数
\[a,2a,3a,\cdots,(p-1)a\pmod p\]
与数
\[1,2,3,\cdots,(p-1)\pmod p\]
相同,尽管它们的次序不同。
\end{lemma}
%
\exercise a)$794=11\cdot72+2$且$9^{72}\equiv1\pmod{73}$,所以
\[9^{794}=9^{11\cdot72+2}={9^{72}}^{11}\cdot9^2\equiv9^2\equiv8\pmod {73}\]
即$a=8$\par
b)由费马小定理$x^{28}\equiv1\pmod{29}$,所以
\[x^{86}=x^{3\cdot28+2}\equiv x^2\equiv6\pmod{29}\]
枚举得$x=8$,$x=21$\par
c)由费马小定理$x^{12}\equiv1\pmod{13}$,所以
\[x^{39}=x^{3\cdot12+3}\equiv x^3\equiv3\pmod{13}\]
枚举发现方程无解。
%
\exercise a)10以内的素数的计算结果:
\begin{center}
\begin{tabular}{c|c}
$p$ & $(p-1)!\pmod p$ \\
\hline
2 & $1\equiv -1$ \\
3 & $2\equiv -1$ \\
5 & $4\equiv -1$ \\
7 & $6\equiv -1$ \\
\end{tabular}
\end{center}
猜测$(p-1)!\equiv 1\pmod p$。(Wilson定理)\par
b)对于$\forall a\in [1, p-1]$,$ax\equiv1\pmod p$在$[1, p-1]$上有解。我们将其称为逆元。考虑是否存在逆元是其本身的$a$,即
\[a^2\equiv 1\pmod p\]
由于$p$整除$a^2-1=(a-1)(a+1)$,所以$a=1$或$a=p-1$。而对于$a\in [2, p-2]$总能将其分为两两一组,其积模$p$余1。于是
\[(p-1)!\equiv 1\cdot(p-1)\equiv -1\pmod p\]
%
\exercise a)对合数进行计算(当然利用了计算机,见code/exe\_9\_3.py):
\begin{center}
\begin{tabular}{c|c}
$m$ & $(m-1)!\pmod m$ \\
\hline
4 & 2 \\
6 & 0 \\
8 & 0 \\
9 & 0 \\
10 & 0 \\
12 & 0 \\
14 & 0 \\
15 & 0 \\
16 & 0 \\
18 & 0 \\
20 & 0 \\
\end{tabular}
\end{center}
合理地猜测,当$m\ge6$时,$(m-1)!\equiv 0\pmod m$。\par
b)显然,如果$(n-1)!\equiv \pmod n$是$-1$,则$n$是素数,否则,如果是0或2,则是合数。其他情况不可能出现。
%
\exercise 费马小定理也可以如下表述:
\[a^{n-1}\not\equiv 1\pmod n\,\Rightarrow n\text{是合数或}n\mid a\,\]
a)能。$1734251=1171\cdot1481$\par
b)能。$64027 = 43\cdot 1489$\par
c)不能。因为$1\mid 2$,不满足费马小定理的条件。事实上$52633=7\cdot 73\cdot 103$。