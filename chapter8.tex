\chapter{同余式}
\begin{theorem}[线性同余定理]
设$a$,$c$与$m$是整数,$m\ge1$,且设$g=\gcd(a,m)$。\par
(a)如果$g\mid c$,则同余式$ax\equiv c\pmod m$没有解。\par
(b)如果$g\nmid c$,则同余式$ax\equiv c\pmod m$恰好有$g$个解。要求这些解,首先求线性方程
\[au+mv=g\]
的一个解$(u_0,v_0)$(见第6章)。则$x_0=cu_0/g$是$ax\equiv c\pmod m$的解,不同余解的完全集由
\[x\equiv x_0+k\cdot\frac{m}{g}\pmod m,\quad k=0,1,2,\cdots,g-1\]
给出。
\end{theorem}
\begin{theorem}[模$p$多项式根定理]
设$p$为素数,
\[f(x)=a_0x^d+a_1x^{d-1}+\cdots+a_d\]
是次数为$d\ge 1$的整系数多项式,且$p$不整除$a_0$,则同余式
\[f(x)\equiv 0\pmod p\]
最多有$d$个模$p$不同余的解。
\end{theorem}
%
\exercise 已知$a_1\equiv b_1\pmod m$及$a_2\equiv b_2\pmod m$。于是存在整数$x$,$y$使得$a_1 - b_1 = mx$,$a_2 - b_2 = my$。于是
\[(a_1+a_2) - (b_1 + b_2) = mx + my = m(x+y)\]
所以$a_1+a_2\equiv b_1+b_2\pmod m$。\par
又有
\[(a_1-a_2) - (b_1 - b_2) = mx - my = m(x-y)\]
所以$a_1+a_2\equiv b_1+b_2\pmod m$。\par
类似地
\begin{align*}
a_1a_2 - b_1b_2 &= (b_1+mx)(b_2+my) - b_1b_2 \\
&= b_1my + b_2mx +m^2xy \\
&= m(b_1y+b_2x +mxy)
\end{align*}
所以$a_1a_2\equiv b_1b_2\pmod m$。
%
\exercise 已知$ac\equiv bc\pmod m$,于是存在唯一整数$x$使得
\[ac - bc = (a-b)c= mx\]
又因为$\gcd(c, m) = 1$,所以$m$整除$a-b$,证毕。
%
\exercise a)$x\equiv 9\pmod {15}$\par
b)无解\par
c)$x\equiv 1,3,5,7\pmod 8$\par
d)$x\equiv 3,4\pmod 7$\par
e)无解\par
%
\exercise a)记$a=b+100c$,其中$b$是$a$的后两位数。模4得到$a\equiv b\pmod4$。所以$a$整除4,当且仅当$b$整除4。\par
b)同上,略。\par
c)记$a=a_1+a_2\cdot10+a_3\cdot100+\cdots$,其中$a_1$,$a_2$,$a_3$,$\cdots$是$a$各个数位上的值。基于$10\equiv 1\pmod3$,得到
\[a\equiv a_1+a_2+a_3+\cdots\pmod3\]
所以$a$被3整除,当且仅当其各位数字之和被3整除。\par
d)同上,略。\par
e)基于$10\equiv -1\pmod{11}$,得到
\[a\equiv a_1-a_2+a_3-a_4+\cdots\pmod{11}\]
%
\exercise a)$6,13\pmod{14}$\par
b)无解。\par
c)$5,18,31,44,57,70,83\pmod{91}$
%
\exercise a)因为$\gcd(72,200)=8$不整除47,所以无解。\par
b)$\gcd(4183, 15087)=47$整除5781,所以有47个解。\par
c)因为$\gcd(1537,6731)=53$不整除2863,所以无解。
%
\exercise 运行code/exe\_7\_7.py得到(程序不会在无解时输出错误信息,而是返回空数组):\par
[]\par
[225, 546, 867, 1188, 1509, 1830, 2151, 2472, 2793, 3114, 3435, 3756, 4077, 4398, 4719, 5040, 5361, 5682, 6003, 6324, 6645, 6966, 7287, 7608, 7929, 8250, 8571, 8892, 9213, 9534, 9855, 10176, 10497, 10818, 11139, 11460, 11781, 12102, 12423, 12744, 13065, 13386, 13707, 14028, 14349, 14670, 14991]\par
[]
%
\exercise 参考code/exe\_7\_8.py:
\begin{center}
\begin{tabular}{c|l}
$m$ & $X$ with $f(X)\equiv0\pmod m$ \\
130 & [2, 47, 67, 112] \\
137 & [99, 104] \\
144 & [] \\
151 & [84, 105] \\
158 & [36, 115] \\
165 & [122, 137, 152] \\
172 & [74, 160] \\
\end{tabular}
\end{center}
%
\exercise a)有两个解$X=1$,$X=9$。\par
b)有4个解$\{1,3,5,7\}$,而这并不与模$p$多项式根定理矛盾,因为8不是素数。
%
\exercise 最多有4个解。\par
\proof 假设$r_1,\cdots,r_5$是5个不同的解。而由于$p$是素数,$x^2-a\equiv0\pmod p$最多有两个不同的解,记为$s_1$,$s_2$。$r_1,\cdots,r_5$必须与$s_1$,$s_2$中的一个模$p$同余。所以其中至少有三个$r_i$模$p$同余。不失一般性,可以假设$r_1\equiv r_2\equiv r_3\pmod p$。再考虑$x^2-a\equiv0\pmod q$最多有两个解,记为$t_1$,$t_2$。同理$r_1$,$r_2$,$r_3$必须与$t_1$,$t_2$中的一个模$p$同余。不失一般性地,可假设$r_1\equiv r_2\pmod q$。于是$r_1\equiv r_2\pmod{pq}$,所以假设不成立,不可能有超过4个解存在。\par
另一方面,可以容易地找到有4个解的例子。比如,如果$p$和$q$是不同的奇素数,则$x^2\equiv 1\pmod{pq}$总是有4个解。如$x^2\equiv 1\pmod{15}$有解1,4,11,14。