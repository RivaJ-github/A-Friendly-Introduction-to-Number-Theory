\chapter{什么是数论}
\exercise
运行code/exe1.py,找到接下来的两个三角平方数:
\begin{verbatim}
35 * 35 = 49 * (49 + 49) = 1225
204 * 204 = 288 * (288 + 288) = 41616
\end{verbatim}
TODO: 阅读第28章,学习如何找到所有三角平方数,用更优雅的方式

\exercise
等差数列求和嘛,前$n$个奇数的和为一个平方数:
\[1+3+5+7+\cdots+\]
几何证明:
\[
    \begin{array}{ccc}
        \begin{array}{cc}
            3 & 3 \\
            1 & 3 \\
        \end{array} &
        \begin{array}{ccc}
            5 & 5 & 5 \\
            3 & 3 & 5 \\
            1 & 3 & 5 \\
        \end{array} &
        \begin{array}{cccc}
            7 & 7 & 7 & 7 \\
            5 & 5 & 5 & 7 \\
            3 & 3 & 5 & 7 \\
            1 & 3 & 5 & 7 \\
        \end{array} \\

        1 + 3 = 4 &
        1 + 3 + 5 = 9 &
        1 + 3 + 5 + 7 = 16 \\
    \end{array}
\]

\exercise
3,5,7是唯一的素数三元组。因为连续三个奇数中,至少有一个是3的倍数。为了使他们都是素数,其中一个必须是3。\par
有两个未被证明的猜想:\par
1、有无穷多个素数$p$,满足$p+2$和$p+6$也是素数;\par
1、有无穷多个素数$p$,满足$p+4$和$p+6$也是素数;

\exercise
由于$N^2-1=(N-1)(N+1)$,当$N\ge3$时,它就不是素数;\par
同理$N^2-4=(N-2)(N+2)$,当$N\ge4$时,它就不是素数;\par
推广到$N^2-a$,如果$a$是一个平方数,则很容易证明没有无穷多个这种形式的素数;\par
反之,如果$a$不是平方数,则普遍认为存在无穷多个这种形式的素数。只是都未得证。

\exercise 
略,对中国玩家来说太简单了

\exercise
a)$M$是三角数当且仅当\underline{$1+8M$}是一个奇平方数;\par
b)$N$是奇平方数当且仅当\underline{$(N-1)/8$}是一个三角数;\par
c)\proof 如果$M$是一个三角数,则$M=m(m+1)/2$,
所以$1+8M=1+4m+4m^2=(1+2m)^2$是一个奇平方数;
相反地,如果$1+8M$是一个奇平方数,则$1+8M=(2k+1)^2$,
所以$M=((2k+1)^2-1)/8=k(k+1)/2$,是一个三角数;\par
如果$N$是奇平方数,$N=(2k+1)^2$,则$(N-1)/8=k(k+1)/2$,是三角数;
如果$(N-1)/8$是三角数,则$(N-1)/8=m(m+1)/2$,$N=(1+2m)^2$是奇平方数。