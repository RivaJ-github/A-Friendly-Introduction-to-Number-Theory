\chapter{佩尔方程}
\begin{theorem}[佩尔方程定理]
设$D$是一个正整数且不是完全平方数,则佩尔方程
\[x^2-Dy^2=1\]
总有正整数解。如果$(x_1,y_1)$是使$x_1$最小的解,则每个解$(x_k,y_k)$可通过取幂得到:
\[x_k+y_k\sqrt D=(x_1+y_1\sqrt D)^k,\quad k=1,2,3,\cdots\]
\end{theorem}
%
\exercise 如果$D$使一个完全平方数,则很容易得到方程$x^2-Dy^2=1$的解。要解$x^2-A^2y^2=1$,可以因式分解:
\[x^2-A^2y^2=(x-Ay)(x+Ay)=1\]
上式成立当且仅当$x-Ay=x+Ay=1$,所以$x=1$,$y=0$。因此,如果$D$是一个完全平方数,则方程$x^2-Dy^2=1$无正整数解。
%
\exercise 查表可知最小解是$(197,42)$,计算
\begin{gather*}
\left(197+42\sqrt{22}\right)^2=77617+16548\sqrt{22} \\
\left(197+42\sqrt{22}\right)^3=30580901+6519870\sqrt{22}
\end{gather*}
所以最小的满足$x>10^6$的解是$(30580901,6519870)$
%
\exercise 假设$(u,v)$是$x^2-11y^2=2$的一个解,其中$u>10$。我们将证明存在另一个解$(s,t)$,其中$s<u$,且
\[u+v\sqrt{11}=(10+3\sqrt{11})(s+t\sqrt{11})\]
然后使用递降法得出结论:$u+v\sqrt{11}=(10+3\sqrt{11})^k$\par
将上式乘积展开得到
\[u+v\sqrt{11}=(10s+33t)+(3s+10t)\sqrt{11}\]
所以
\[10s+33t=u,\qquad 3s+10t=v\]
解二元方程组:
\[s=10u-33v,\qquad t=-3u+10v\]
容易验证$(s,t)$是$x^2-y^2=1$的一个解:
\[s^2-11t^2=(10u-33v)^2-11(-3u+10v)^2=u^2-11v^2=1\]
还需验证$s$和$t$是正的,且$s<u$。\par
首先验证$s>0$,由
\[u^2=1+11v^2>11v^2,\quad u>\sqrt{11}v\]
因此
\[s=10u-33v>(10\sqrt11-33)v>0\]
接着验证$t>0$:
\begin{align*}
u&>10 && \text{根据假设} \\
u^2&>100 && \\
100u^2&>100+99u^2 && \\
100u^2-100&>99u^2 && \\
u^2-1&>\frac{99}{100}u^2 && \\
11v^2&>\frac{99}{100}u^2 && \text{由于}u^2-11v^2=1 \\
v&>\frac{3}{10}u &&
\end{align*}
所以$t$是正的。最后,由于$s$和$t$是正的,且$u=10s+33t$,显然$s<u$。证毕。
%
\exercise 设$T_n$是第$n$个三角数,$S_n$是第$n$个平方数,$P_n$是第$n$个五角数。我们有
\[T_n=\frac{n(n+1)}{2},\quad S_n=n^2, P_n=\frac{3n^2-n}{2}\]
a)五角三角数是方程$P_n=T_m$的解,即
\[\frac{3n^2-n}{2}=\frac{m(m+1)}{2}\]
两边乘以2,并配方得到
\[3\left(n-\frac{1}{6}\right)^2-\frac{1}{12}=\left(m-\frac{1}{12}\right)-\frac{1}{4}\]
两边乘以36,整理得
\[(6m+3)^2-3(6n-1)^2=6\]
令$u=6m+3$,$v=6n-1$,我们需要解
\[u^2-3v^2=6\]
这是个类佩尔方程。有一个明显的解是$(u,v)=(3,-1)$对应$m=n=0$。另外佩尔方程$x^2-3y^2=1$有最小解$(x,y)=(2,1)$。所以我们可以通过计算
\[u_k+v_k\sqrt3=(3-\sqrt3)(2+\sqrt3)^k\]
得到$u^2-3v^2=6$的解。
\begin{center}
    \begin{tabular}{lll}
        $(u_1,v_1)=(3,1)$ & $m=0$ & $n=1/3$ \\
        $(u_2,v_2)=(9,5)$ & $m=1$ & $n=1$ \\
        $(u_3,v_3)=(33,19)$ & $m=5$ & $n=10/3$ \\
        $(u_4,v_4)=(123,71)$ & $m=20$ & $n=12$ \\
        $(u_5,v_5)=(459,265)$ & $m=76$ & $n=133/3$ \\
        $(u_6,v_6)=(1713,989)$ & $m=285$ & $n=165$ \\
        $(u_7,v_7)=(6393,3691)$ & $m=1065$ & $n=1846/3$ \\
        $(u_8,v_8)=(23859,13775)$ & $m=3976$ & $n=2296$ \\
    \end{tabular}
\end{center}
当然,我们对$m$,$n$的整数解感兴趣。所以我们得到五角三角数$1,210,40655,7906276,\cdots$,有无穷多个。\par
b)类似地,解$P_n=S_m$,我们得到佩尔方程:
\[x^2-6y^2=1\]
其中$x=6n-1$,$y=2m$。该佩尔方程的最小解是$(5,2)$,所以所有解可通过$5+2\sqrt6$的幂得到。下面是前几个解:
\begin{center}
    \begin{tabular}{lll}
        $(x_1,y_1)=(5,2)$ & $m=1$ & $n=1$ \\
        $(x_2,y_2)=(49,20)$ & $m=25/3$ & $n=10$ \\
        $(x_3,y_3)=(485,198)$ & $m=81$ & $n=99$ \\
        $(x_4,y_4)=(4801,1960)$ & $m=2401/3$ & $n=980$ \\
        $(x_5,y_5)=(47525,19402)$ & $m=7921$ & $n=9701$ \\
        $(x_6,y_6)=(470449,192060)$ & $m=235225/3$ & $n=96030$ \\
        $(x_7,y_7)=(4656965,1901198)$ & $m=776161$ & $n=950599$ \\
        $(x_8,y_8)=(46099201,18819920)$ & $m=23049601/3$ & $n=9409960$ \\
    \end{tabular}
\end{center}
前几个平方五角数是$1,9801,94109401,903638458801$,也有无穷多个。\par
c)既是三角数又是平方数也是五角数的数满足:
\[\frac{l(l+1)}{2}=m^2=\frac{3n^2-n}{2}\]
它可能只有一个解$l=m=n=1$,但是可能还没有人证明。