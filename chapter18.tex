\chapter{幂、根与不可破密码}
\exercise 信息是161528231130,翻译后是"Fermat"
%
\exercise a、b)练习17.4证明了只要$m$是不同素数的乘积,算法都能正常工作。\par
c)令$k=5$,则$a^k=3^5\equiv9\pmod{18}$,所以$b=9$。由$5k-4\phi(m)=1$,计算$b^k=9^5\equiv9\pmod{18}$。我们没有得到原始信息$a=3$。
%
\exercise 略
%
\exercise 运行code/exe\_18\_4.py,得到如下原文(当然已经进行了一些美化):\par
a)Mathematics is the queen of science, and number theory is the queen of mathematics. K.F.Gauss
b)前提是得到$m$得分解,$p=123456789012345681631$,$q=7746289204980135457$(靠计算机分解并不现实):\par
The different branches of arithmetic, replied the Mock Turtle: Ambition, Distraction, Uglification, and Derision.
%
\exercise 略,参考上题使用的代码,并考虑以下补足:
\begin{itemize}
\item 假设仅考虑ASCII码,最大为3位数,需要考虑如何区分两位和三位的情况。如12\,234,122\,34。这个问题可能同时存在于原文和编码。
\end{itemize}
%
\exercise 略
%
\exercise 略