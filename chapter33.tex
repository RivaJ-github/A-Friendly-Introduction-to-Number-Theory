\chapter{丢番图逼近}
\begin{theorem}[狄利克雷的丢番图逼近定理---版本1]
假设$D$是一个非完全平方数的正整数,则存在无穷多个正整数对$(x,y)$使得
\[|x-y\sqrt D|<1/y\]
\end{theorem}
\begin{theorem}[狄利克雷的丢番图逼近定理---版本2]
假设$\alpha$是一个无理数,即$\alpha$是一个不能表成分数$a/b$的实数,则存在无穷多个正整数对$(x,y)$使得
\[|x-y\alpha|<1/y\]
\end{theorem}
%
\exercise 复制狄利克雷的丢番图逼近定理---版本1的证明过程,将其中的$\sqrt D$替换为$\alpha$,即得证。
%
\exercise 相关代码见code/exe\_32\_2.py\par
a)对于$y\le20$,最接近$\gamma$的分数是$x/y=21/13\approx1.615384615384615\dots$\par
b)\begin{center}
\begin{tabular}{|c|c|c|c|}
\hline
$x$ & $y$ & $|x-y\gamma|\cdot y$ & $x/y$ \\
\hline 
\hline
$55$ & $34$ & $0.447291$ & $1.6176470588$ \\
\hline
$89$ & $55$ & $0.447184$ & $1.6181818182$ \\
\hline
$144$ & $89$ & $0.447225$ & $1.6179775281$ \\
\hline
$233$ & $144$ & $0.447209$ & $1.6180555556$ \\
\hline
$377$ & $233$ & $0.447215$ & $1.6180257511$ \\
\hline
$610$ & $377$ & $0.447213$ & $1.6180371353$ \\
\hline
$987$ & $610$ & $0.447214$ & $1.6180327869$ \\
\hline
$1597$ & $987$ & $0.447214$ & $1.6180344478$ \\
\hline
\end{tabular}
\end{center}
c)略
%
\exercise a)$r_1=1,r_2=2,r_3=3/2,r_4=5/3,r_5=8/5,r_6=13/8,r_7=21/13,r_8=34/21,r_9=55/34,r_{10}=89/55$\par
b)使用近似值$\gamma=1.61803$和a)得到的值:
\begin{center}
\begin{tabular}{|c|c|c|}
\hline
$|r_1-\gamma|=0.61803$ & $|r_2-\gamma|=0.38197$ & $|r_3-\gamma|=0.11803$ \\
\hline 
$|r_4-\gamma|=0.04863$ & $|r_5-\gamma|=0.01803$ & $|r_6-\gamma|=0.00697$ \\
\hline
$|r_7-\gamma|=0.00265$ & $|r_8-\gamma|=0.00101$ & $|r_9-\gamma|=0.00039$ \\
\hline
 & $|r_{10}-\gamma|=0.00015$  \\
\hline
\end{tabular}
\end{center}
看起来差值迅速地变小,所以$r_n$是对$\gamma$对一个很好的有理逼近。\par
c)
\[r_{20}=10946/6765,\quad r_{30}=1346269/832040,\quad r_{40}=165580141/102334155\]
偏差如下:
\begin{gather*}
|r_{20}-\gamma|=0.0000000097719083935743356\\
 |r_{30}-\gamma|=0.0000000000006459277557269\\
 |r_{40}-\gamma|=0.0000000000000000000000000
\end{gather*}
d)当$n\rightarrow\infty$,$r_n,r_{n-1}\rightarrow r$,于是$r=1+1/r$,即$r^2-r-1=0$,解得$r=(1\pm\sqrt{5})/2$,排除负根,得到$r=(1+\sqrt{5})/2=r$\par
e)使用归纳法很容易证明分子分母都是斐波那契数。
%
\exercise 运行code/exe33\_4.py,建立下表:
\begin{center}
\begin{tabular}{ccccc}
\hline
$x$ & $y$ & $|x-y\sqrt2|$ & $y\cdot |x-y\sqrt2|$ & $y^2\cdot |x-y\sqrt2|$ \\
\hline
17 & 12 & 0.02943725 & 0.35324702 & 4.24 \\
24 & 17 & 0.04163056 & 0.70771953 & 12.03 \\
41 & 29 & 0.01219331 & 0.35360596 & 10.25 \\
58 & 41 & 0.01724394 & 0.70700165 & 28.99 \\
99 & 70 & 0.00505063 & 0.35354437 & 24.75 \\
140 & 99 & 0.00714267 & 0.70712482 & 70.01 \\
239 & 169 & 0.00209204 & 0.35355494 & 59.75 \\
338 & 239 & 0.00295859 & 0.70710369 & 169.00 \\
577 & 408 & 0.00086655 & 0.35355313 & 144.25 \\
816 & 577 & 0.00122549 & 0.70710731 & 408.00 \\
1393 & 985 & 0.00035894 & 0.35355344 & 348.25 \\
1970 & 1393 & 0.00050761 & 0.70710669 & 985.00 \\
3363 & 2378 & 0.00014868 & 0.35355338 & 840.75 \\
4756 & 3363 & 0.00021026 & 0.70710680 & 2378.00 \\
\hline
\end{tabular}
\end{center}
看起来$|x-y\sqrt2|$的值比$1/y$略小,但它并不能和$1/y^2$一样小。$|x-y\sqrt2|$的值看起来符合$c/y$的形式,其中$c$的值在$c_1\approx 0.353553\approx 1/\sqrt8$和$c_2\approx 0.707106\approx 1/\sqrt2$之间来回变换。以上观察引出了下面的定理:如果$C$是任意大于$1/\sqrt8$的数,则存在无穷多个正整数对$(x,y)$,使得$|x-y\sqrt2|<C/y$。而如果条件替换为$C<1/\sqrt8$,结论正相反。\par
b)(\romannumeral1)$|x^2-2y^2|$是非负整数,所以它是0或$\ge1$。如果$x^2-2y^2=0$,则$(x/y)^2=2$,而这是不可能的,因为$\sqrt2$是无理数。\par
(\romannumeral2)$|x^2-2y^2|<1/y$意味着$x<y\sqrt2+1/y$。因此$x+y\sqrt2<2y\sqrt2+1/y$,结合(\romannumeral1)的结论,得到下式:
\[1\le |x^2-2y^2|=|x-y\sqrt2|\cdot|x+y\sqrt2|<|x-y\sqrt2|\cdot\left(2y\sqrt2+1/y\right)\]
两边同时除以$2y\sqrt2+1/y$即得证。\par
当$y$很大时,几乎可以忽略$1/y$,所以$|x-y\sqrt2|$不能比$1/2y\sqrt2=1/y\sqrt8$小太多。这就解释了a)中的数据。