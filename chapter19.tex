\chapter{素性测试与卡米歇尔数}
\begin{theorem}[卡米歇尔数的考塞特判别法]
设$n$是合数,则$n$是卡米歇尔数当且仅当它是奇数,且整除$n$的每个素数$p$满足下述两个条件:
\begin{enumerate}
\item $p^2$不整除$n$;
\item $p-1$不整除$n-1$。
\end{enumerate}
\end{theorem}
\begin{theorem}[素数的一个性质]
设$p$是奇素数,记
\[p-1=2^kq,\quad q\text{是奇数}\]
设$a$是不被$p$整除的整数。则下述两个条件之一成立:
\begin{enumerate}
\item $a^q$模$p$余1;
\item 数$a^q$,$a^{2q}$,$a^{2^2q}$,$\cdots$,$a^{2^{k-1}q}$之一模$p$余$-1$。
\end{enumerate}
\end{theorem}
\begin{theorem}[合数的拉宾--米勒测试]
设$n$是奇素数,记$n-1=2^kq$,$q$是奇数。对不被$n$整除对某个$a$,如果下述两个条件都成立,则$n$是合数。
\begin{enumerate}
\item $a^q\not\equiv1\pmod n$;
\item 对所有$i=0,1,2,\cdots,k-1$,$a^{2^iq}\not\equiv-1\pmod n$。
\end{enumerate}
\end{theorem}
\hbox{\vrule
  \vbox{\hrule
    \hbox {\hfil
      \vbox spread 8pt{\vfil
        如果$n$是奇合数,则1与$n-1$之间至少有75\%的数可作为$n$对拉宾-米勒证据。
      \vfil}
    \hfil}
  \hrule}
\vrule}
%
\exercise a)$n$是卡米歇尔数,所以$g^n\equiv g\pmod n$,所以$g^n\equiv g\pmod p$。接着将$n$写成如下形式:
\[n=(p-1)k+j\quad \text{其中}0\le j\le p-2\]
由费马小定理:
\[g^n=(g^{p-1})^k\cdot g^j\equiv g^j\pmod p\]
所以$g^i\equiv g\pmod p$。但是$g$是模$p$的原根,$1,g,g^2,\cdots,g^{p-2}$模$p$不同余,所以$j=1$。这证明了
\[n=(p-1)k+1\]
因此$p-1$整除$n-1$。\par
b)令$m=n/p$,写成$m=(p-1)u+v$(其中$0\le v\le p-2$)。仿照a)的计算过程:
\begin{align*}
g^n=g^{pm}&=(g^p)^{(p-1)u+v} \\
&\equiv g^v\pmod p &\text{由于} g^p\equiv g\pmod p \\
&& \text{和}g^{p-1}\equiv 1\pmod p
\end{align*}
同理可得$v=1$,$(n/p) - 1 = m-1=(p-1)u$
%
\exercise 不存在仅有两个素因数的卡米歇尔数。可以使用考塞特判别法证明,但是用练习19.1 b)的结论更容易证明。设$n=pq$,其中$p$和$q$都是奇素数。判别法要求$p-1$整除$(n/p)-1$,即$p-1$整除$q-1$,同理$q-1$整除$p-1$。这意味着$p=q$,$n=p^2$,但这与条件一矛盾,证毕。
%
\exercise a)$1105=5\cdot13\cdot17$,是卡米歇尔数。\par
b)$1235=5\cdot13\cdot19$,不是卡米歇尔数。\par
c)$2821=7\cdot13\cdot31$,是卡米歇尔数。\par
d)$6601=7\cdot23\cdot41$,是卡米歇尔数。\par
e)$8911=7\cdot19\cdot67$,是卡米歇尔数。\par
f)$10659=3\cdot11\cdot17\cdot19$,不是卡米歇尔数。\par
g)$19747=7\cdot13\cdot31$,不是卡米歇尔数。\par
h)$105545=5\cdot11\cdot19\cdot101$,不是卡米歇尔数。\par
i)$126217=7\cdot13\cdot19\cdot73$,是卡米歇尔数。\par
j)$162401=17\cdot41\cdot233$,是卡米歇尔数。\par
k)$172081=7\cdot13\cdot31\cdot61$,是卡米歇尔数。\par
l)$188461=7\cdot13\cdot19\cdot109$,是卡米歇尔数。
%
\exercise a)\proof $n-1=(6k+1)(12k+1)(18k+1)-1=1296k^3+396k^2+36k=36k(36k^2+11k+1)$。$n$被$6k$,$12k$,$18k$整除,所以$n$是卡米歇尔数。\par
b)运行code/exe\_19\_4.py,得到下表
\begin{center}
\begin{tabular}{c|c}
$k$ & $n$ \\
\hline
1 & 1729 \\
6 & 294409 \\
35 & 56052361 \\
45 & 118901521 \\
51 & 172947529 \\
\end{tabular}
\end{center}
%
\exercise 略
%
\exercise a)这个算法不是很高效,但是正确的
\begin{lstlisting}
def isCarmichael(n):
    '''判断n是否是卡米歇尔数'''
    factors = factoringPrimeFactors(n)
    if len(factors) == 1:
        return False
    for factor in factors:
        if (factor[1] > 1) or (n-1) % (factor[0] - 1) != 0:
            return False
    return True
\end{lstlisting}
b、c)运行code/exe\_19\_6.py得到100000以内的卡米歇尔数有:
\[561, 1105, 1729, 2465, 2821, 6601, 8911, 10585, 15841, 29341, 41041, 46657, 52633, 62745, 63973, 75361\]
大于1000000的最小卡米歇尔数是:1024651
%
\exercise a)
\begin{align*}
2^{69}&\equiv -138\pmod{1105} \\
2^{2\cdot69}&\equiv 259\pmod{1105} \\
2^{4\cdot69}&\equiv -324\pmod{1105} \\
2^{8\cdot69}&\equiv 1\pmod{1105}
\end{align*}
b)$n-1=299408=2^3\cdot36801$
\begin{align*}
2^{36801}&\equiv 512\pmod{299409} \\
2^{2\cdot36801}&\equiv -32265\pmod{299409} \\
2^{4\cdot36801}&\equiv 1\pmod{299409}
\end{align*}
$294409=37\cdot73\cdot109$
\begin{align*}
    294408&=36\cdot8178 \\
    294408&=72\cdot4089 \\
    294408&=108\cdot2726
\end{align*}
c))$n-1=118901520=2^4\cdot7431345$
\begin{align*}
2^{7431345}&\equiv 45274074\pmod{118901521} \\
2^{2\cdot7431345}&\equiv 1758249\pmod{118901521} \\
2^{4\cdot7431345}&\equiv 1\pmod{118901521} 
2^{8\cdot7431345}&\equiv 1\pmod{118901521} \\
\end{align*}
$118901521=271\cdot541\cdot811$
\begin{align*}
    118901520&=270\cdot 438750 \\
    118901520&=540\cdot 220188 \\
    118901520&=810\cdot 146792
\end{align*}
%
\exercise 
\begin{lstlisting}
def RabinMillerTest(n, times = 10):
    '''
    合数的拉宾-米勒测试
    @param n: 待测试合数
    @param times: 测试次数
    @return: 是否是合数(True则一定是合数,False则大概率是合数)
    '''
    k = 0
    q = n - 1
    while (q % 2 == 0):
        q //= 2
        k += 1
    for i in range (0, times):
        a = random.randint(2, n-2)
        ss = successive_square(a, q, n)
        if (ss == 1 or ss == n - 1): 
            return False
        for j in range(1, k):
            ss = ss * 2 // n
            if (ss == -1):
                return False
    return True
\end{lstlisting}
根据测试b、c)是合数,a、d)大概率是素数