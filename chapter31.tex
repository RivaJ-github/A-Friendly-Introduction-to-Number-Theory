\chapter{再论三角平方数}
\begin{theorem}[三角平方数定理]
\begin{enumerate}
\item 方程
\[x^2-2y^2=1\]
的每个正整数解都可通过将$3+2\sqrt2$自乘得到,即解$(x_k, y_k)$可以通过展开下式得到:
\[x_k+y_k\sqrt2=(3+2\sqrt2)^k,\quad k=1,2,3,\cdots\]
\item 每个三角平方数$n^2=\frac{1}{2}m(m+1)$由
\[m=\frac{x_k-1}{2},\quad n=\frac{y_k}{2},\quad k=1,2,3,\cdots\]
给出,其中$(x_k,y_k)$是由(1)得到的解。
\end{enumerate}
\end{theorem}
%
\exercise 最小的解是$(9, 4)$。通过自乘$9+4\sqrt5$得到如下解:
\[(161, 72),(2889, 1292),(51841, 23184),(930249, 416020),(16692641, 7465176)\]
%
\exercise a、b)假设$A$可以表成两个三角数的和,即
\[\frac{m(m+1)}{2}+\frac{n(n+1)}{2}=A\]\par
两边乘以8,并做简单的代数变换:
\begin{align*}
4m(m+1)+4n(n+1)&=8A \\
(2m+1)^2+(2n+1)^2=2(4A+1)
\end{align*}
因此$A$能表成两个三角数之和,当且仅当$2(4A+1)$能表成两个平方数之和。根据两平方数之和定理,$4A+1$能分解为$p_1p_2\cdots p_rM^2$,其中$p_i$模4与1。\par
c)任何一个数都能被表示为不超过三个三角数的和,但并不容易证明。类似地,任何一个数都能被表示为不超过四个平方数之和;任何数都能被表示为不超过五个五边形数之和。
%
\exercise a)如果$(x_k,y_k)$是方程$x^2-2y^2=1$的解,根据三角平方数定理:
\[(3+2\sqrt2)(x_k+y_k\sqrt2)=(3x_k+4y_k)+(2x_0+3y_0)\sqrt2\]
由此得到递推式:
\[x_{k+1}=3x_k+4y_k,\quad y_{k+1}=2x_k+3y_k\]\par
b)根据关系式$m=(x-1)/2$,$n=y/2$。如果$(m,n)$给出一个三角平方数,那么下一对$(m',n')$应满足
\begin{align*}
m'&=\frac{x'-1}{2}=\frac{3x+4y-1}{2}=\frac{3(2m+1)+4(2n)-1}{2} \\
&=1+3m+4n \\
n'&=\frac{y'}{2}=\frac{2x+3y}{2}=\frac{2(2m+1)+3(2n)}{2}=1+2m+3n
\end{align*}
所以答案是$(1+3m+4n,1+2m+3n)$。\par
c)设$(m,n)$给出三角平方数$L$,即$L=n^2=(m^2+m)/2$。于是
\[m=(-1+\sqrt{1+8L})/2\]
由a)可知,下一个三角平方数满足:
\begin{align*}
n'^2&=(1+2m+3n)^2=\left(1+2\frac{-1+\sqrt{1+8L}}{2}+3\sqrt{L}\right)\\
&=1+17L+6\sqrt{L+8L^2}
\end{align*}
%
\exercise a)略\par
b)观察发现第$n+1$个图形会比第$n$个图形多三条含有$n+1$个顶点的边,但是重复计算了两个顶点。所以得到递推式如下:
\[P_{n+1}=P_n+3n-2,\quad P_1=1\]
所以$P_n=1+4+7+\cdots + (3(n-1) -2) + (3n-2)$,这是一个等差数列,所以$P_n=\frac{3n^2-n}{2}$。\par
c)利用b)的公式,很容易得到$P_{10}=145$,$P_{100}=14950$
