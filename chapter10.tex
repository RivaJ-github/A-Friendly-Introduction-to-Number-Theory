\chapter{同余式、幂与欧拉公式}
\begin{theorem}[欧拉公式]
如果$\gcd(a, m) = 1$,则
\[a^{\phi(m)}\equiv 1\pmod m\]
其中$\phi(m)$是1与$m$之间与$m$互素的整数个数。记为$\phi(m)=\#\{a:1\le a \le m,\,\gcd(a,m)=1\}$(欧拉函数)。
\end{theorem}
\begin{lemma}
如果$\gcd(a, m) = 1$,则数列
\[b_1a,b2_a, b_3a,\cdots,b_{\phi(m)}a\pmod m\]
与数列
\[b_1,b2_, b_3,\cdots,b_{\phi(m)}\pmod m\]
相同,尽管它们可能次序不同。
\end{lemma}
%
\exercise a)因为$\gcd(b_i,m)=1$,根据线性同余定理,恰好存在一个模$m$的解。所以对于任意$b_i$,存在一个$b_j$使得$b_ib_j\equiv 1\pmod m$,如果$i\not=j$则,可将$b_i$和$b_j$从$B$中消去。所以同余式$B$可以表示为满足$b_i^2\equiv1\pmod m$的$b_i$。令$c_1,c_2,\cdots,c_r$为满足该性质的$b_i$。\par
接下来考虑同余式$c_ix\equiv-1\pmod m$。再次利用线性同余定理,存在一个模$m$的解,假设是$x=d$。同余式两边平方,且已知$c_i^2\equiv 1\pmod m$,得到$d^2\equiv 1\pmod m$,所以$d$必须是某个$c_j$,且一定不是$c_i$。通过这种方式每个$c_i$与不同的$c_j$组成一对,使得$c_ic_j\equiv -1\pmod{m}$。这意味着我们可以消去$c_ic_j$,代之以因子$-1$。最终我们将消去$B$中所有的$b_i$,剩下一堆$-1$的乘积。因此$B\equiv \pm 1\pmod{m}$\par
b)下表是$2\le m\le20$的例子:
\begin{center}
\begin{tabular}{c||c|c|c|c|c|c|c|c|c|c|}
$m$ & & $2$ & $3$ & $4$ & $5$ & $6$ & $7$ & $8$ & $9$ & $10$ \\
\hline
$B\pmod m$ &  & $-1$ & $-1$ & $-1$ & $-1$ & $-1$ & $-1$ & $1$ & $-1$ & $-1$ \\
\hline
\hline
$m$ & 11 & $12$ & $13$ & $14$ & $15$ & $16$ & $17$ & $18$ & $19$ & $20$ \\
\hline
$B\pmod m$ & $-1$ & $1$ & $-1$ & $-1$ & $1$ & $1$ & $-1$ & $-1$ & $-1$ & $1$ \\
\hline
\end{tabular}
\end{center}
a)中证明了$B\equiv (-1)^t\pmod m$,其中$t$是$x^2\equiv 1\pmod m$的解的个数的一半。实时证明当且仅当模$m$有原根时$B\equiv -1\pmod m$,否则$B\equiv 1\pmod m$。有论文证明模$m$有原根,当且仅当$m=2$,$m=4$,$m=p^k$,$m=2p^k$,其中$p$是奇素数,$k\ge1$
%
\exercise 根据欧拉公式,$7^{\phi(3750)}=7^{1000}\equiv1\pmod{3750}$,所以$a\equiv7^{3003}\equiv(7^3=343)\pmod{3750}$。又因为$1\le a\le5000$,所以$a=343$或$a=343+3750=4093$。又因为$7\nmid a$所以$a=4093$。
%
\exercise a)对于任意$a$($\gcd(a, 561)=1$)。根据费马小定理
\[a^2\equiv1\pmod3\quad a^{10}\equiv1\pmod{11}\quad a^{16}\equiv1\pmod{17}\]
因为560被2,10,16整除。所以
\[a^{560}=(a^2)^{280}\equiv 1^{280}\equiv1\pmod 3\]
\[a^{560}=(a^{10})^{56}\equiv 1^{56}\equiv1\pmod{11}\]
\[a^{560}=(a^{16})^{35}\equiv 1^{35}\equiv1\pmod{17}\]
这意味着$a^{560}-1$同时被2,11,17整除,所以也被561整除。因此$a^{560}\equiv 1\pmod{561}$。\par
b)接下来的两个米歇尔数是$1105=5\cdot13\cdot17$和$1729=7\cdot13\cdot19$。一般而言,不同素数的乘积$m=p_1p_2\cdots p_r$,如果满足$p_i-1$整除$m-1$,则$m$是一个米歇尔数。存在无限个米歇尔数,但是它的证明非常困难。