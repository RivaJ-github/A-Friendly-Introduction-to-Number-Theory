\chapter{二次互反律}
\begin{theorem}[二次互反律]
设$p$,$q$是不同的奇素数,则
\begin{align*}
\left(\frac{-1}{p}\right)&=
\begin{cases}
    1&\text{当$p\equiv1\pmod4$时}\\
    -1&\text{当$p\equiv3\pmod4$时}\\
\end{cases} \\
\left(\frac{2}{p}\right)&=
\begin{cases}
    1&\text{当$p\equiv1$或$7\pmod8$时}\\
    -1&\text{当$p\equiv3$或$5\pmod8$时}\\
\end{cases} \\
\left(\frac{q}{p}\right)&=
\begin{cases}
    \left(\frac{p}{q}\right)&\text{当$p\equiv1\pmod4$或$q\equiv1\pmod4$时}\\
    -\left(\frac{p}{q}\right)&\text{当$p\equiv3\pmod4$且$q\equiv3\pmod4$时}\\
\end{cases} 
\end{align*}
\end{theorem}
\begin{theorem}[广义二次互反律]
设$a$,$b$为正奇数,则
\begin{align*}
\left(\frac{-1}{b}\right)&=
\begin{cases}
    1&\text{当$b\equiv1\bmod4$时}\\
    -1&\text{当$b\equiv3\bmod4$时}\\
\end{cases} \\
\left(\frac{2}{b}\right)&=
\begin{cases}
    1&\text{当$b\equiv1$或$7\pmod8$时}\\
    -1&\text{当$b\equiv3$或$5\pmod8$时}\\
\end{cases} \\
\left(\frac{a}{b}\right)&=
\begin{cases}
    \left(\frac{b}{a}\right)&\text{当$a\equiv1\pmod4$或$b\equiv1\pmod4$时}\\
    -\left(\frac{b}{a}\right)&\text{当$a\equiv b\equiv 3\pmod4$时}\\
\end{cases} 
\end{align*}
\end{theorem}
%
\exercise a)$1$;b)$-1$;c)$1$;d)$-1$
%
\exercise 由原式得$(2x-3)^2\equiv13\pmod{31957}$。用二次互反律求$(\frac{13}{31957})=1$,所以存在解$a$使得$a^2\equiv13\pmod{31957}$。对应原式的解为$(3\pm a)/2\pmod{31957}$。(不必担心$a$是偶数的情况,因为我们总可以加上对$a$加上31957使其变为奇数)。
%
\exercise \proof 设$p_1,p_2,\cdots,p_r$是模3余1的素数列表。构造数$A=(2p_1p_2\cdots p_r)^2+3$,并将其分解为$A=q_1q_2\cdots q_s$。注意$A\equiv3\pmod4$,所以$q_i$是奇数。并且其中至少有一个是模4余3的。可以重排$q_i$使得$q_1$是模4余3的。因为任意$p_i$不整除$A$,所以$q_1$不等于任意$p_i$,当然也不等于2或3,因为2和3不整除$A$。下面证明$q_1$是模3余1的。\par
因为$A\equiv 0\pmod q$,所以$x=2p_1p_2\cdots p_r$是$x^2\equiv-3\pmod{q_1}$的解。这意味着$(\frac{-3}{p})=1$。利用二次互反律 (注意$q_1\equiv3\pmod4$)
\[\left(\frac{-3}{q_1}\right)=\left(\frac{-1}{q_1}\right)\left(\frac{3}{q_1}\right)=-1\times-\left(\frac{q_1}{3}\right)=\left(\frac{q_1}{3}\right)\]
现在我们知道$(\frac{q_1}{3})=1$。另外,显然有$(\frac{1}{3})=1$和$(\frac{2}{3})=-1$,所以必须有$q_1\equiv1\pmod3$。因此$q_1$是一个新的模3余1的素数。
%
\exercise \proof 因为$p$整除$A^2-5$,所以$A^2-5\equiv0\pmod p\Rightarrow A^2\equiv5\pmod p$。所以5是模$p$的二次剩余,$\left(\frac{5}{p}\right)=1$。因为$5\equiv1\pmod4$,所以根据二次互反律,$\left(\frac{p}{5}\right)=1$,但是5的二次剩余是1或4,所以$p$模5余1或4。
%
\exercise 下面的代码仅通过了本章涉及的例子:
\begin{lstlisting}
def JacobiSymbol(a, b):
    '''计算雅可比符号(a/b)'''
    # 忽略参数校验,假设初始参数都是正奇数
    if (a == 2):
        return 1 if b % 8 == 1 or b % 8 == 7 else -1

    flag = False    # 去除因子2
    while (a % 2 == 0):
        a //= 2
        flag = not flag

    res = JacobiSymbol(2, b) if flag else 1

    if (a == 1):
        return res

    return res * JacobiSymbol(b % a, a)
\end{lstlisting}
%
\exercise a)设$b=p_1p_2\cdots p_rq_1q_2\cdots q_s$,其中$p_i\equiv1\text{或}7\pmod8$,$q_i\equiv3\text{或}5\pmod8$。$p_ip_j$和$q_iq_j$模8余1或7,$p_iq_j$模8余3或5。所以
\[b\equiv
\begin{cases}
1 \text{或}7\pmod8 & \text{当$s$是偶数}\\
3 \text{或}5\pmod8 & \text{当$s$是奇数}
\end{cases}\]
根据雅可比符号的定义:
\[\left(\frac{2}{b}\right)=\left(\frac{2}{p_1}\right)\left(\frac{2}{p_2}\right)\cdots\left(\frac{2}{p_r}\right)\left(\frac{2}{q_1}\right)\left(\frac{2}{q_2}\right)\cdots\left(\frac{2}{q_s}\right)\]
根据原版的二次互反律$\left(\frac{2}{p_i}\right)=1$,$\left(\frac{2}{q_i}\right)=-1$,所以:
\[\left(\frac{2}{b}\right)=(-1)^s=
\begin{cases}
1 & \text{当$s$是偶数}\\
-1 & \text{当$s$是奇数}
\end{cases}\]
b)\proof 设$b=p_1p_2\cdots p_rq_1q_2\cdots q_s$,其中$p_i\equiv1\pmod4$,$q_i\equiv3\pmod4$。则当$s$是偶数时,$b\equiv1\pmod4$,当$s$是奇数时,$b\equiv3\pmod4$。同理,分解$a=p'_1p'_2\cdots p'_uq'_1q'_2\cdots q'_v$,当$v$是偶数时,$a\equiv1\pmod4$,当$v$是奇数时,$a\equiv3\pmod4$。计算下式:
\[\left(\frac{a}{b}\right)=\prod\left(\frac{p'_i}{p_j}\right)\prod\left(\frac{p'_i}{q_j}\right)\prod\left(\frac{q'_i}{p_j}\right)\prod\left(\frac{q'_i}{q_j}\right)\]
原版二次互反律告诉我们$\left(\frac{p'_i}{p_j}\right)$,$\left(\frac{p'_i}{q_j}\right)$,$\left(\frac{q'_i}{p_j}\right)$都等于1。因此
\begin{align*}
\left(\frac{a}{b}\right)=(-1)^{sv}&=
\begin{cases}
1 & \text{$sv$是偶数}\\
-1 & \text{$sv$是奇数}
\end{cases} \\
&=\begin{cases}
1 & \text{$s$或$v$是偶数}\\
-1 & \text{$s$和$v$是奇数}
\end{cases} \\
&=\begin{cases}
1 & \text{$a$或$b$模4余1}\\
-1 & \text{$a$和$b$模4余3}
\end{cases}
\end{align*}
证毕。
%
\exercise a)由欧拉准则,
\[a^{(p-1)/2}\equiv\left(\frac{a}{p}\right)\pmod p\]
已知$a$是模$p$得一个二次剩余,$\left(\frac{a}{p}\right)=1$。因此,
\[\left(a^{(p+1)/4}\right)^2=a^{(p+1)/2}=a^{(p-1)/2}\cdot a=a\equiv\pmod p\]
b)由a)得到其中一个解是:$x = 7^{(787+1)/4} = 7^{197}$,用逐次平方法可得$7^{197}\equiv 105\pmod{787}$,即得解$x\equiv 105\pmod{787}$
%
\exercise a)\proof 由欧拉准则$a^{(p-1)/2}\equiv\left(\frac{a}{p}\right)\equiv1\pmod p$,所以$a^{(p-1)/4}\equiv\pm1\pmod p$,如果值是$+1$,则
\[\left(a^{(p+3)/8}\right)^2=a^{(p+3)/4}=a\cdot a^{(p-1)/4}\equiv a\pmod p\]
此时,$x=a^{(p+3)/8}$是同余式$x^2\equiv a\pmod p$的解。\par
如果值是$-1$,则
\[\left(2a\cdot (4a)^{(p-5)/8}\right)^2=4a^2\cdot (4a)^{(p-5)/4}=a\cdot 2^{(p-1)/2}\cdot a^{(p-1)/4}\]
我们假设了$a^{(p-1)/4}\equiv-1\pmod p$。另外由于$p\equiv5\pmod 8$,根据二次互反律,2不是二次剩余,即$2^{(p-1)/2}\equiv-1\pmod p$。因此
\[\left(2a\cdot (4a)^{(p-5)/8}\right)^2\equiv 1\pmod p\]
此时,$x=2a\cdot (4a)^{(p-5)/8}$是同余式$x^2\equiv a\pmod p$的解。\par
b)用逐次平方法计算:
\[5^{(541-1)/4}=5^{135}\equiv1\pmod{541}\]
所以其中一个解为
\[x=5^{(541+3)/8}=5^{68}\equiv345\pmod{541}\]
(另一个解是$x\equiv 196\pmod{541}$)\par
c)用逐次平方法计算:
\[13^{(653-1)/4}=13^{163}\equiv-1\pmod{653}\]
所以其中一个解为
\[x=2\cdot13\cdot(4\cdot13)^{(653-3)/8}=26\cdot52^{81}\equiv288\pmod{653}\]
(另一个解是$x\equiv 365\pmod{653}$)
%
\exercise 见code/exe\_22\_9.py
\begin{lstlisting}
def solution(a, p):
    if JacobiSymbol(a, p) != 1:
        raise ValueError('a is not a quadratic residue of p')
    if successive_square(a, (p - 1) // 4, p) == 1:
        return successive_square(a, (p + 3) // 8, p)
    else:
        return 2 * a * successive_square(4 * a, (p - 5) // 8, p) % p
\end{lstlisting}
\[494^2\equiv 17\pmod{1021},\quad 163^2\equiv 23\pmod{1021}\]
31不是1021的二次剩余,所以$x^2\equiv 31\pmod{1021}$无解。
%
\exercise a)使用二次互反律得到
\[\left(\frac{11}{1729}\right)=\left(\frac{1729}{11}\right)=\left(\frac{2}{11}\right)=-1\]
而$11^{864}\equiv 1\pmod{1729}$他们并不相等。所以$1729$不是素数。(事实上$1729=7\cdot13\cdot19$。\par
b)$2^{(1293337 - 1)/2}\equiv429596\pmod{1293337}$,由于即不是1也不是$-1$,所以根据欧拉准则可以断定他不是素数。虽然$2^{1293336}\equiv1\pmod{1293337}$。