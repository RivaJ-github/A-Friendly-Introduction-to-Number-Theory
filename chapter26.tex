\chapter{像1,2,3一样简单}
%
\exercise a、b、c)略
d)首先$A_1=B_1=1$,所以$B_1$整除$1!$成立。接着假设命题对$n$成立。则
\begin{align*}
\frac{A_{n+1}}{B_{n+1}}&=\frac{A_n}{B_n}+\frac{1}{n+1}\\
&=\frac{(n+1)A_n+B_n}{(n+1)B_n}
\end{align*}
以上分式可能不是最简的,但是$B_{n+1}$一定整除$(n+1)B_n$,而归纳假设$B_n$整除$n!$,所以$B_{n+1}$整除$(n+1)n!=(n+1)!$。证毕。
%
\exercise $n=1$时,$F_1=1$,$F_3-1=2-1=1=F_1$。假设等式对于$n$成立,则
\[F_1+F_2+\cdots+F_n+F_n+1=(F_{n+2}-1)+F_{n+1}=F_{n+3}-1\]
即对于$n+1$也成立。证毕。
%
\exercise 当$n=6$时,$6!=720$,$\frac{6^6}{2^6}=729$,不等式成立。假设已知不等式对于某个$n\ge6$成立,则
\[\frac{n!2^n}{n^n}\le 1\]
则
\begin{align*}
\frac{(n+1)!2^{n+1}}{(n+1)^{n+1}}&=\frac{n!2^{n+1}}{(n+1)^n} \\
&=2\cdot\frac{n!2^{n}}{(n+1)^n} \\
&=2\cdot\frac{n!2^{n}}{n^n}\cdot\frac{n^n}{(n+1)^n} \\
&\le 2\cdot\frac{n^n}{(n+1)^n} \\
&=2\cdot \left(\frac{n}{n+1}\right)^n
\end{align*}
接着考察函数$f(x)=(1+1/x)^x$,由于$f'(x)=\ln xf(x)$,所以$f(x)$在$x>1$时单调递增。所以对任意$x\ge1$,$f(x)\ge f(1)=2$。这意味着
\[\left(1+\frac{1}{n}\right)^n\ge 2\qquad \left(\frac{n}{n+1}\right)^n\le \frac{1}{2}\]
所以
\[\frac{(n+1)!2^{n+1}}{(n+1)^{n+1}}\le2\cdot \left(\frac{n}{n+1}\right)^n\le1\]
即原命题对$n+1$也成立,证毕。
%
\exercise (题目写错了,应该是$F(x)=x^2-x+41$)\par
a)$F(1),F(22),\cdots,F(40)$都是素数。\par
b)但是$F(41)$就不是素数。更一般的结论是:一个(非常数)多项式不可能只取素数值。首先找到某个$n_0$使得$|f(n_0)\ge2$。令$D=f(n_0)$,则$f(n_0+kD)$一定被$D$整除。这是因为
\[f(n_0+kD)\equiv f(n_0)=D\equiv0\pmod D\]
由于$k\rightarrow \infty$时,$|f(n_0+kD)|\rightarrow \infty$,所以它不总是等于$D$。所以它有因子$D$,是个合数。
%
\exercise 归纳证明的错误出现在$n=2$时。