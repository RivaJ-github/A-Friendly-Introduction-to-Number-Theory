\chapter{幂模$p$与原根}
$a$模$p$的次数(或阶)指($a$与$p$互素):
\[e_p(a)=\text{(使得$a^e\equiv1\pmod p$的最小指数$e\ge1$)}\]
\begin{theorem}[次数整除性质]
设$a$是不被素数$p$整除的整数,假设$a^n\equiv 1\pmod p$,则次数$e_p(a)$整除$n$,特别地,次数$e_p(a)$总整除$p-1$。
\end{theorem}
\hbox{\vrule
  \vbox{\hrule
    \hbox {\hfil
      \vbox spread 8pt{\vfil
        具有最高次数$e_p(g)=p-1$的数$g$称为模$p$的原根。
      \vfil}
    \hfil}
  \hrule}
\vrule}
\begin{theorem}[原根定理]
每个素数$p$都有原根,更精确地,有恰好$\phi(p-1)$个原根。
\end{theorem}
\begin{conjecture}[阿廷猜想]
有无穷多个素数$p$使得2是模$p$的原根。
\end{conjecture}
\begin{conjecture}[广义阿廷猜想]
设整数$a$不是完全平方也不等于$-1$,则有无穷多个素数$p$使得$a$是模$p$的原根。
\end{conjecture}
%
\exercise a)$1+2+3+\cdots+(p-1)=(p-1)p/2$,所以只要$p\not=2$,它模$p$余0。\par
b)$1^2+2^2+3^2+\cdots+(p-1)^2=(p-1)p(2p-1)/6$,只要$p\not=2,3$,它模$p$余0。\par
c)如果$k$被$p-1$整除,则有费马小定理,和式的每一项都为1,所以和式模$p$余$-1$。否则,设$g$是模$p$的原根。则
\[1^k+2^k+3^k+\cdots+(p-1)^k\equiv g^k+g^{2k}+g^{3k}+g^{(p-1)k}\pmod p\]
因为$1,g,g^2,\cdots,g^{p-1}$模$p$与$1,2,\cdots,p-1$是相同的(顺序可能不同)。于是
\[1+g^k+g^{2k}+g^{3k}+g^{(p-2)k}\equiv\frac{1-g^{(p-1)k}}{1-g^k}\pmod p\]
(注意,因为假设$k$不被$p-1$整除,所以$g^k\not\equiv1\pmod p$。)但$g^{p-1}\equiv1\pmod p$,所以和式模$p$余0。综上:
\[1^k+2^k+3^k+\cdots+(p-1)^k\equiv 
\begin{cases}
0\pmod p & \text{如果$p-1$不整除$k$} \\
-1\pmod p & \text{如果$p-1$整除$k$} 
\end{cases}\]
%
\exercise a)(\romannumeral1)$e_9(2)=6$;(\romannumeral2)$e_9(2)=4$;(\romannumeral3)$e_9(2)=4$;(\romannumeral4)$e_9(2)=4$\par
b)我们知道$2^{\phi(m)}\equiv1\pmod m$。我们还知道$2^{e_m}\equiv1\pmod m$。设$g=\gcd(e_m, \phi(m))$,并设$e_mu-\phi(m)v=g$。由此,我们可以计算:
\begin{align*}
2^{e_mu}&\equiv2^{\phi(m)v+g}\pmod m\\
(2^{e_m})^u&\equiv(2^{\phi(m)})^v\cdot2^g\pmod m \\
1^u&\equiv 1^v\cdot2^g\pmod m \\
1&\equiv 2^g\pmod m
\end{align*}
但是$e_m$是满足$a^e\equiv1\pmod m$的最小指数,所以必须有$e_m\le g=\gcd(e_m,\phi(m))$。所以必须有$e_m=g$。而$g$整除$\phi(m)$,所以$e_m$整除$\phi(m)$。(虽然证明的是$a=2$的情况,但证明过程使用于任意$a$)
%
\exercise a)$e_{11}=10,\quad
e_{13}=12,\quad
e_{15}=4,\quad
e_{17}=8,\quad
e_{19}=18$\par
b)对于$\gcd(m,n)=1$,$e_{mn}=e_me_n/\gcd(e_m,e_n)$\par
c)表中$e_{103}=51$,$e_{109}=36$,$\gcd(e_{103},e_{109})=3$。于是$e_{11227}=51\cdot36/3=512$\par
d)设$E=e_me_n/\gcd(e_m,e_n)$,我们需要证明$E$是满足$2^E\equiv1\pmod m$的最小值。首先,已知$2^{e_m}\equiv1\pmod m$,两边求$e_n/gcd(e_m,e_n)$次幂,得到$2^E\equiv1\pmod m$。同理可得$2^E\equiv1\pmod n$。换句话说,$2^E-1$同时被$m$和$n$整除。由于$m$,$n$互素,所以$2^E\equiv 1\pmod{mn}$。\par
下面假设$K\ge1$满足$2^K\equiv1\pmod{mn}$。证明$K\ge E$,也就证明了$E$是最小的。由$2^K\equiv1\pmod{m}$,$2^K\equiv1\pmod{n}$,可知$e_m\mid K$,$e_n\mid K$。据此不难得到$e_me_n/gcd(e_m,e_n)\mid K$。证毕。\par
e)从表格中观察,似乎$e_{p^k}=p^{k-1}e_p$。由此$e_{68921}=e_{41^3}=41^2e_{41}=41^2\cdot20=33620$。\par
f)公式$e_{p^k}=p^{k-1}e_p$对于某些素数是错误的,如$p=1093,3511$。对于这两个素数,$e_{p^2}=e_p$,$e_{p^k}=p^{k-2}e_p$。已验证小于$4\cdot10^{12}$的素数中只有这两个素数有$e_{p^k}=p^{k-1}e_p$。
% 
\exercise 模13的原根为:$\{2,6,7,11\}$
\begin{align*}
\{a:e_{13}(a)=1\}&=\{1\}\\
\{a:e_{13}(a)=2\}&=\{12\}\\
\{a:e_{13}(a)=3\}&=\{3,9\}\\
\{a:e_{13}(a)=4\}&=\{5,8\}\\
\{a:e_{13}(a)=6\}&=\{4,10\}\\
\{a:e_{13}(a)=12\}&=\{2,6,7,11\}\\
\end{align*}
%
\exercise a)只有$g^5$和$g^7$是模37的原根。\par
b)$g^k$是模$p$的原根,当且仅当$\gcd(k, p-1)=1$。\par
\proof 必要性:设$G=\gcd(k, p-1)$,如果$G>1$,则$(g^k)^{(p-1)/G}=(g^{p-1})^{k/G}\equiv1^{k/G}\equiv1\pmod p$。所以$g^k$不是模$p$的原根,因为它的$((p-1)/G)$次方模$p$余1。\par
充分性:接着假设$(g^k)^n\equiv1\pmod p$,且$G=1$。这意味着$ku-(p-1)v=1$有整数解。于是$g^{ku}=(g^{p-1})^vg\equiv g\pmod p$。两边做$n$次方,得$1\equiv g^{kun}\equiv g^n\pmod p$,所以$p-1$整除$n$。这意味着使得$(g^k)^n$模$p$余1的最小指数是$p-1$,即$g^k$是模$p$的原根。证毕。\par
c)$p-1=21168=2^4\cdot3^3\cdot7^2$。所以$k$只需不被3或5或7整除。所以$g^5,g^{11},g^{13},g^{17},g^{19}$是模21169的原根。
%
\exercise 100以内有原根3的素数有:5, 7, 17, 19, 29, 31, 43, 53, 79, 89
%
\exercise (请去掉题目中的$a=p^2$中的$=p^2$。)\par
记$a=b^2$,则
\[a^{(p-1)/2}=(b^2)^{(p-1)/2}=b^{p-1}\equiv1\pmod p\]
所以$e_p(a)\le(p-1)/2$,于是$a$不可能是模$p$的原根。
%
\exercise a)\[g, g^2,g^3,\cdots, g^{p-2}, g^{p-1}\]
给出了所有模$p$的非零值。由$(g^k)^2=g^{2k}$,所以偶数次方肯定是二次剩余。假设某个奇数次方也是二次剩余,则$g^{2k+1}\equiv c^2\pmod p$,两边做$(p-1)/2$次方得到
\[1\equiv c^{p-1}\equiv (c^2)^{(p-1)/2}\equiv g^{(2k+1)(p-1)/2}\equiv g^{k(p-1)}g^{(p-1)/2}\equiv g^{(p-1)/2}\pmod p\]
这与$g$是原根矛盾。\par
b)由a)可知$(\frac{g^n}{p}=(-1)^n)$,所以
\[\left(\frac{g^n}{p}\right)\left(\frac{g^m}{p}\right)=(-1)^n\cdot(-1^m)=(-1)^{n+m}=\left(\frac{g^{n+m}}{p}\right)\left(\frac{g^ng^m}{p}\right)\]
c)总能选择$n$,使得$a\equiv g^n\pmod p$,则:
\begin{align*}
\left(\frac{a}{p}\right)=1&\Longleftrightarrow n\text{是偶数,不妨记为}n=2k \\
&\Longrightarrow a^{(p-1)/2}\equiv g^{2k(p-1)/2}\equiv(g^{p-1})^k\equiv 1\pmod p
\end{align*}
类似地,
\begin{align*}
\left(\frac{a}{p}\right)=-1&\Longleftrightarrow n\text{是奇数,不妨记为}n=2k+1 \\
&\Longrightarrow a^{(p-1)/2}\equiv g^{(2k+1)(p-1)/2}\equiv(g^{p-1})^kg^{(p-1)/2}\equiv g^{(p-1)/2}\pmod p
\end{align*}
我们知道$g^{(p-1)/2}$的平方模$p$余1,所以$g^{(p-1)/2}$与1或$-1$同余,但是$g$是原根,所以它不能同余1,只能同余$-1$。
%
\exercise $e_p(2)$整除$p-1$。而$p-1=2q$,所以$e_p(2)$是1或2或$q$或$2q$。又$2^{e_p(2)}\equiv1\pmod p$,所以$e_p(2)$不是1或2(因为$p=2q+1$且$q$至少为5,所以$p$至少为11)。所以$e_p(2)$是$q$或$2q$。首先观察$2^q\pmod p$,由于$q=(p-1)/2$,所以根据欧拉准则$2^q\equiv(\frac{2}{p})\pmod p$。另一方面,由于$p=2q+1$且$q\equiv1\pmod4$,所以$p\equiv3\pmod8$。由二次互反律,$\left(\frac{2}{p}\right)=-1$,所以$2^q\equiv-1\pmod p$。所以$e_p(2)\not=q$,所以仅剩的可能性是$e_p(2)=2q=p-1$。所以2是模$p$的原根。
%
\exercise 设$g$是模$p$的原根,并设$b\equiv g^u\pmod p$。则$b$的$k$次根(假设存在)$g^v\pmod p$满足$vk\equiv u\pmod{p-1}$。假设这个同余式有解,我们知道它有$\gcd(k, p-1)$个解,所以$b$有$\gcd(k, p-1)$个模$p$的$k$次根。
%
\exercise 
\begin{lstlisting}
def epa(p, a):
    '''
    计算使得$a^e=1(mod p)$的最小正整数e
    '''
    for e in range(1, p // 2 + 1):
        if (successive_square(a, e, p) == 1):
            return e
    return p - 1
\end{lstlisting}
%
\exercise 下面的程序给出$p$的所有原根:
\begin{lstlisting}
def primitive_root(p):
    '''
    计算p的primitive root
    '''
    res = []
    for a in range(1, p):
        for r in range(1, p):
            if (r == p - 1):
                res.append(a)
            elif (successive_square(a, r, p) == 1):
                break
    return res
\end{lstlisting}
%
\exercise $e_{mn}(a)=e_m(a)e_n(a)/\gcd(e_m(a),e_n(a))=\mathrm{LCM}(e_m(a),e_n(a))$(证明过程见练习28.3)
%
\exercise a)8,12,15,16,20,21,24,28,30,32,33,35,36,39,40,42,44,45,48没有原根。\par
b)$m$有原根,当且仅当$m=p^k$或$m=2p^k$,其中$p$是任意奇素数。\par
c)\proof 必要性:假设$m$没有以上形式,则$m$可被分解为
\[m=m_1m_2,\text{其中}\gcd(m_1,m_2)=1\text{且}m_1,m_2\ge3\]
假设$g$是$m$的原根。则$e_m(g)=\phi(m)=\phi(m_1)\phi(m_2)$。但前面的练习告诉我们
\[e_m(g)=e_{m_1}(g)e_{m_2}(g)/\gcd(e_{m_1}(g),e_{m_2}(g))\]
其中$e_{m_1}(g)\le\phi(m_1)$,$e_{m_2}(g)\le\phi(m_2)$。所以要使$e_m(g)=\phi(m)$必须有$e_{m_1}=\phi(m_1)$,$e_{m_2}=\phi(m_2)$且$\gcd(\phi(m_1),\phi(m_2))=1$。但由于$m_1,m_2\ge3$,所以$\phi(m_1)$,$\phi(m_2)$都是偶数,他们不互素。所以如果$m$不具有形式$p^k$或$2p^k$,则$m$没有原根。\par
充分性:设$m=p^k$,$g$是模$p$的原根。则总可以构造$g+ap$为模$p^2$的原根。在加上一个二次项成为$p^3$的原根,以此类推,可以构造任意的$p^k$的原根。至于$2p^k$,由于$\phi(p^k)=\phi(2p^k)$,所以$p^k$的原根也是$2p^k$的原根。
%
\exercise a)有$N!$个$N\times N$的转置矩阵\par
b)$A^6$是个单位矩阵。\par
c)考虑任意向量$e=[e_1, e_2, e_3,\cdots, e_N]^T$,$Ae$的结果是$[e_i]$的另一个排列。而最多只有$N!$种排列,所以$A^{N!}$总是单位矩阵。更进一步,假设$n_j$是使得$A^{n_j}e$的第$j$个元素回到初始位置的最小次数。则$A^{\mathrm{LCM}(n_1,n_i,\cdots,n_N)}$是单位矩阵。\par
d)我们需要构造两个循环,如:
\[A=\begin{bmatrix}
0 & 1 & 0 & 0 & 0 \\
1 & 0 & 0 & 0 & 0 \\
0 & 0 & 0 & 1 & 0 \\
0 & 0 & 0 & 0 & 1 \\
0 & 0 & 1 & 0 & 0 
\end{bmatrix}\]
矩阵$A$会交换$e_1,e_2$,需要两步回到原位($n_1=n_2=2$),同时交换$e_3,e_4,e_5$,需要三步回到原位($n_3=n_4=n_5=3$)。因此需要$A^6$将所有向量回归原位。
%
\exercise a)
\[\begin{bmatrix}
1 & 0 & 0 \\ 
0 & 0 & 1 \\  
0 & 1 & 0 
\end{bmatrix}\quad
\begin{bmatrix}
0 & 1 & 0 \\ 
1 & 0 & 0 \\  
0 & 0 & 1 
\end{bmatrix}\quad
\begin{bmatrix}
0 & 1 & 0 \\ 
0 & 0 & 1 \\ 
1 & 0 & 0  
\end{bmatrix}\quad
\begin{bmatrix}
0 & 0 & 1 \\ 
1 & 0 & 0 \\ 
0 & 1 & 0  
\end{bmatrix}\]
b、c、d)略
%
\exercise 3是17的原根,建表如下:
\begin{center}
\begin{tabular}{c*{16}{|c}}
\hline
第$i$行 & 1 & 2 & 3 & 4 & 5 & 6 & 7 & 8 & 9 & 10 & 11 & 12 & 13 & 14 & 15 & 16 \\
\hline
第$i$列 & 3 & 9 & 10 & 13 & 5 & 15 & 11 & 16 & 14 & 8 & 7 & 4 & 12 & 2 & 6 & 1 \\
\hline
\end{tabular}
\end{center}
%
\exercise a)提示:$g_1^i$是$[1,p-2]\rightarrow[2,p-1]$的一个双射,$g_2^j$也是。\par
b)略,类似正文的Welch构造的证明过程。\par
c)略。