\chapter{勾股数组与单位圆}
\begin{theorem}
圆$x^2+y^2=1$上的坐标是有理数的点都可由公式
\[(x,y)=(\frac{1-m^2}{1+m^2}, \frac{2m}{1+m^2})\]
得到,其中$m$取有理数值,(点$(-1,0)$例外,这是$m\rightarrow \infty$值时的极限)
\end{theorem}
%
\exercise
a)如果$u$和$v$有公因数$d$,则$a$,$b$,$c$有公因数$d^2$\par
b)令$u = 3 \ge v = 1 > 0$,$(u^2-v^2, 2uv, u^2+v^2) = (8, 6, 10)$不是本原的。
存在四种可能性\par
c)运行code/exe3\_1\_c:
\begin{table}[!h]
    \tiny\setlength{\tabcolsep}{0.5em}
    \begin{tabular}{|c|c|c|c|c|c|c|c|c|c|}
        \hline
       \diagbox{$u$}{$v$} & 1 & 2 & 3 & 4 & 5 & 6 & 7 & 8 & 9 \\
        \hline
        2 & \cellcolor{red!20}(3, 4, 5) & & & & & & & & \\
        \hline
        3 & (8, 6, 10) & \cellcolor{red!20}(5, 12, 13)& & & & & & &  \\
        \hline
        4 & \cellcolor{red!20}(15, 8, 17) & (12, 16, 20) & \cellcolor{red!20}(7, 24, 25) & & & & & & \\
        \hline
        5 & (24, 10, 26) & \cellcolor{red!20}(21, 20, 29) & (16, 30, 34) & \cellcolor{red!20}(9, 40, 41) & & & & &  \\
        \hline
        6 & \cellcolor{red!20}(35, 12, 37) & (32, 24, 40) & (27, 36, 45) & (20, 48, 52) & \cellcolor{red!20}(11, 60, 61) & & & &  \\
        \hline
        7 & (48, 14, 50) & \cellcolor{red!20}(45, 28, 53) & (40, 42, 58) & \cellcolor{red!20}(33, 56, 65) & (24, 70, 74) & \cellcolor{red!20}(13, 84, 85) & & &  \\
        \hline
        8 & \cellcolor{red!20}(63, 16, 65) & (60, 32, 68) & \cellcolor{red!20}(55, 48, 73) & (48, 64, 80) & \cellcolor{red!20}(39, 80, 89) & (28, 96, 100) & \cellcolor{red!20}(15, 112, 113) & &  \\
        \hline
        9 & (80, 18, 82) & \cellcolor{red!20}(77, 36, 85) & (72, 54, 90) & \cellcolor{red!20}(65, 72, 97) & (56, 90, 106) & (45, 108, 117) & (32, 126, 130) & \cellcolor{red!20}(17, 144, 145) & \\
        \hline
        10 & \cellcolor{red!20}(99, 20, 101) & (96, 40, 104) & \cellcolor{red!20}(91, 60, 109) & (84, 80, 116) & (75, 100, 125) & (64, 120, 136) & \cellcolor{red!20}(51, 140, 149) & (36, 160, 164) & \cellcolor{red!20}(19, 180, 181)  \\       
        \hline
    \end{tabular}
\end{table}\par
d)$(u^2-v^2,2uv, u^2+v^2)$是本原的,当且仅当$u>v$且$u$,$v$没有公因子,且其中一个是偶数\par
e)\proof 必要性:如果$u$,$v$都是奇数,则$u^2-v^2,2uv, u^2+v^2$都是偶数,则三元组不是本原的。另外,由a),$u$,$v$必须互素。因为三元组必须为正数,所以$u>v$。\par
充分性:假设$(u^2-v^2,2uv, u^2+v^2)$不是本原的,则存在$d\ge2$整除这三个数。同时整除以下两个数:
\[(u^2-v^2) + (u^2+v^2) = 2u^2\quad \text{及}\quad 
(u^2+v^2) - (u^2+v^2) = 2v^2\]
故$d=2$或$d$同时整除$u$和$v$。后者与条件$u$和$v$没有公因子矛盾。
考虑$d=2$的情况,$u^2-v^2$是偶数。则$u$,$v$是同奇偶性的,他们不能同时偶数,否则就有公因子2。
于是$u$,$v$都是奇数,但这与条件矛盾。因此假设不成立。\par
证毕(虽然题目只要求证明充分性,但必要性显然容易证明)
%
\exercise a)连立圆方程和过点(1,1)的直线方程
\[
\begin{cases}
    x^2+y^2=2 \\
    y-1=m(x-1)
\end{cases}\Rightarrow \begin{cases}
    x=\frac{m^2-2m-1}{m^2+1} \\
    y=\frac{-m^2-2m+1}{m^2+1}
\end{cases}
\]
b)$x^2+y^2=3$上不存在有理点以开始推导
%
\exercise
\[(x,y) = (\frac{1+m^2}{1-m^2}, \frac{2m}{1-m^2})\]
%
\exercise $(433/211, -9765/1331)$,TODO: 见第41章讨论为什么第三个点是有理数
%
\exercise a)令平方三角数$T_n=\frac{1}{2}(n^2+n)=m^2$,
则$8m^2=4n^2+4n=(2n+1)^2-1$,因此$(2n+1)^2-2(2m)^2=1$。所以
$x^2-2y^2=1$的$x$为奇数,$y$为偶数的解可以表示每个平方三角数。\par
b)$(\frac{2m^2+1}{2m^2-1},\frac{2m}{2m^2-1})$\par
c)将$m=v/u$带入,得到令一点的坐标为:
\[(x,y)=\left(\frac{2v^2+u^2}{2v^2-u^2}, \frac{2vu}{2v^2-u^2}\right)\]
令$2v^2-u^2=1$,则坐标(改变符号)为$(2v^2+u^2,2vu)$\par
d)运行code/exe\_3\_5\_d.py得到接下来的三个整数解,制表如下:
\begin{table}[!h]
\centering
\begin{tabular}{|c|c|c|c|c|}
    $x$ & $y$ & $n$ & $m$ & $m^2$ \\
    3 & 2 & 2 & 1 & 1 \\
    17 & 12 & 16 & 6 & 36 \\
    577 & 408 & 576 & 204 & 41616 \\
    665857 & 470832 & 665856 & 235416 & 55420693056 \\
\end{tabular}
\end{table}\par
e)假设我们从一个整数解$(x_0, y_0)$开始,则下一个解的纵坐标是$2y_0x_0 > y_0$,所以每次迭代得到的整数解都是新的点。\par
f)TODO: 第29、30章将研究形如$x^2-Dy^2=1$的方程。

