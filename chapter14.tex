\chapter{梅森素数}
\begin{statement}
如果对整数$a\le 2$与$n\le2$,$a^n-1$是素数,则$a$必等于2且$n$一定是素数。
\end{statement}
%
\exercise 注意当$m$为奇数时,$x^m+1$被$x+1$整除。假设$n$不是2的幂,则可以将其分解为$n=2^km$,其中$m$是不小于3的奇数。则
\[a^n+1=\left(a^{2^k}\right)^m+1\]
能被$a^{2^k}+1$整除。所以$a^n+1$不是素数。
%
\exercise 不妨设$k>m$,则$F_k-2=2^{2^k}+1-2=2^{2^k}-1=(2^{2^m})^{2^{k-m}}-1$,所以$F_k-2$被$2^{2^m}$即$F_m$整除。接着假设$d$是$F_m$和$F_k$的公因子。由于$d$整除$F_m$,而$F_m$整除$F_k-2$,所以$d$也整除$F_k-2$。所以$d$也整除2。但是$d\not=2$,因为$F_k$和$F_m$是奇数。所以$d$必须是1。所以$\gcd(F_k,F_m)=1$
%
\exercise a)$n=7$时得到下一个素数,$(3^7-1)/2=1093$\par
b)设$n=2m$,则
\[\frac{3^n-1}{2}=\frac{9^m-1}{2}\]
由于$9^m\equiv1\pmod8$,所以$(9^m-1)/2$被4整除。\par
c)设$n=5m$,则
\[\frac{3^n-1}{2}=\frac{243^m-1}{2}\]
注意$243-1=242=2\cdot11^2$,所以
\[243^m=(2\cdot11^2+1)^m\equiv1\pmod(11^2)\]
所以$(243^m-1)/2$被$11^2$整除。\par
d)未知。
