\chapter{幂模m与逐次平方法}
\begin{algorithm}[逐次平方计算$a^k\pmod m$]
用下述步骤计算$a^k\pmod m$的值:\par
\begin{enumerate}
\item 将$k$表成2的幂次和:
\[k=u_0+u_1\cdot2+u_2\cdot2^2+\cdots+u_r\cdot2^r\]
其中每个$u_i$是0或1。(这种表示式叫做$k$的二进制展开)
\item 使用逐次平方法制作模$m$的$a$的幂次表
\begin{align*}
a^1&\equiv A_0\pmod m\\
a^2&\equiv(a^1)^2\equiv A_0^2\equiv A_1\pmod m\\
a^4&\equiv(a^2)^2\equiv A_1^2\equiv A_2\pmod m\\
a^8&\equiv(a^4)^2\equiv A_2^2\equiv A_3\pmod m\\
\vdots\\
a^{2^r}&\equiv(a^{2^{r-1}})^2\equiv A_{r-1}^2\equiv A_r\pmod m
\end{align*}
注意要计算表的每一行,仅需要取前一行最末的数,平方它然后用模$m$简化。也注意到表有$r+1$行,其中$r$是第1步中$k$的二进制展开中2的最高指数。
\item 乘积
\[A_0^{u_0}\cdot A_1^{u_1}\cdot A_2^{u_2}\cdots A_r^{u_r}\pmod m\]
同余于$a^k\pmod m$。注意到所有$u_i$是0或1,因此这个数实际上是$u_i$等于1的那些$A_i$的乘积。
\end{enumerate}
\end{algorithm}
%
\exercise a)二进制展开$13=8+4+1$,制作幂次表
\begin{align*}
5^1&\equiv 5\pmod{23}\\
5^2&\equiv 25 \equiv 2\pmod{23}\\
5^4&\equiv2^2\equiv 4\pmod{23}\\
5^8&\equiv4^2\equiv 16\pmod{23}
\end{align*}
$5^{13}=5^1\cdot5^4\cdot5^8\equiv 16\cdot4\cdot5\equiv16\cdot5\equiv21\pmod{23}$\par
b)二进制展开$749=512+128+64+32+8+4+1$,制作幂次表
\begin{align*}
28^1&= 28&\pmod{1147}\\
28^2&\equiv 28^2\equiv784&\pmod{1147}\\
28^4&\equiv 784^2\equiv 1011&\pmod{1147}\\
28^8&\equiv 1011^2\equiv 144&\pmod{1147}\\
28^{16}&\equiv 144^2\equiv 90&\pmod{1147}\\
28^{32}&\equiv 90^2\equiv 71&\pmod{1147}\\
28^{64}&\equiv 71^2\equiv 453&\pmod{1147}\\
28^{128}&\equiv 453^2\equiv 1043&\pmod{1147}\\
28^{256}&\equiv 1043^2\equiv 493&\pmod{1147}\\
28^{512}&\equiv 493^2\equiv 1032&\pmod{1147}
\end{align*}
\begin{align*}
28^{749}&=28^{512}\cdot28^{128}\cdot28^{64}\cdot28^{32}\cdot28^{8}\cdot28^{4}\cdot28^{1} \\
&\equiv 1032\cdot1043\cdot453\cdot71\cdot144\cdot1011\cdot28 \\
&\equiv 289\pmod{1147}
\end{align*}
%
\exercise a)略\par
b)
\begin{lstlisting}
''' a^k(mod m) '''
def successive_square(a, k, m):
    b = 1
    while k >= 1:
        if k % 2 == 1:
            b = (b * a) % m
        a = a * a % m
        k = k // 2
    return b
\end{lstlisting}
c)(\romannumeral1)$2^{1000}\equiv562\pmod{2379}$\par(\romannumeral2)$567^{1234}\equiv3214\pmod{4321}$\par
(\romannumeral3)$47^{258008}\equiv1296608\pmod{1315171}$
%
\exercise a)$7^{7386}\equiv702\pmod{7387}$,由费马小定理可知,它不是素数($7387=83\cdot89$)。\par
b)$7^{7392}\equiv1\pmod{7393}$,可以推断它大概率是素数,但不能肯定(虽然它正好是素数)。
%
\exercise 略
%
\exercise $2^{9990}\equiv3362\pmod{9991}$,所以9991一定是合数,事实上$9991=97\cdot103$